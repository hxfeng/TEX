\documentclass[11pt,a4paper]{article}
\usepackage[BoldFont,SlantFont]{xeCJK}
\setCJKmainfont[BoldFont=SimHei]{SimSun}
\setCJKmonofont{SimSun}% 设置缺省中文字体
\parindent 2em   %段首缩进
 \usepackage[includeheadfoot]{geometry}%布局设置
\geometry{left=2.5cm,right=1.5cm,top=2.5cm,bottom=1.5cm, footskip=0.5cm}%页面设置
%%%%%%%%%%%%%%%%%%%%%%%%%%%%%字体设置%%%%%%%%%%%%%%%%%%%%%%%

%-----------------------xeCJK下设置中文字体------------------------------%  
\setCJKfamilyfont{song}{SimSun}                             %宋体 song  
\newcommand{\song}{\CJKfamily{song}}                        % 宋体   (Windows自带simsun.ttf)  
\setCJKfamilyfont{xs}{NSimSun}                              %新宋体 xs  
\newcommand{\xs}{\CJKfamily{xs}}  
\setCJKfamilyfont{fs}{FangSong_GB2312}                      %仿宋2312 fs  
\newcommand{\fs}{\CJKfamily{fs}}                            %仿宋体 (Windows自带simfs.ttf)  
\setCJKfamilyfont{kai}{KaiTi_GB2312}                        %楷体2312  kai  
\newcommand{\kai}{\CJKfamily{kai}}                            
\setCJKfamilyfont{yh}{Microsoft YaHei}                    %微软雅黑 yh  
\newcommand{\yh}{\CJKfamily{yh}}  
\setCJKfamilyfont{hei}{SimHei}                                    %黑体  hei  
\newcommand{\hei}{\CJKfamily{hei}}                          % 黑体   (Windows自带simhei.ttf)  
\setCJKfamilyfont{msunicode}{Arial Unicode MS}            %Arial Unicode MS: msunicode  
\newcommand{\msunicode}{\CJKfamily{msunicode}}  
\setCJKfamilyfont{li}{LiSu}                                            %隶书  li  
\newcommand{\li}{\CJKfamily{li}}  
\setCJKfamilyfont{yy}{YouYuan}                             %幼圆  yy  
\newcommand{\yy}{\CJKfamily{yy}}  
\setCJKfamilyfont{xm}{MingLiU}                                        %细明体  xm  
\newcommand{\xm}{\CJKfamily{xm}}  
\setCJKfamilyfont{xxm}{PMingLiU}                             %新细明体  xxm  
\newcommand{\xxm}{\CJKfamily{xxm}}    
\setCJKfamilyfont{hwsong}{STSong}                            %华文宋体  hwsong  
\newcommand{\hwsong}{\CJKfamily{hwsong}}  
\setCJKfamilyfont{hwzs}{STZhongsong}                        %华文中宋  hwzs  
\newcommand{\hwzs}{\CJKfamily{hwzs}}  
\setCJKfamilyfont{hwfs}{STFangsong}                            %华文仿宋  hwfs  
\newcommand{\hwfs}{\CJKfamily{hwfs}}  
\setCJKfamilyfont{hwxh}{STXihei}                                %华文细黑  hwxh  
\newcommand{\hwxh}{\CJKfamily{hwxh}}  
\setCJKfamilyfont{hwl}{STLiti}                                        %华文隶书  hwl  
\newcommand{\hwl}{\CJKfamily{hwl}}  
\setCJKfamilyfont{hwxw}{STXinwei}                                %华文新魏  hwxw  
\newcommand{\hwxw}{\CJKfamily{hwxw}}  
\setCJKfamilyfont{hwk}{STKaiti}                                    %华文楷体  hwk  
\newcommand{\hwk}{\CJKfamily{hwk}}  
\setCJKfamilyfont{hwxk}{STXingkai}                            %华文行楷  hwxk  
\newcommand{\hwxk}{\CJKfamily{hwxk}}  
\setCJKfamilyfont{hwcy}{STCaiyun}                                 %华文彩云 hwcy  
\newcommand{\hwcy}{\CJKfamily{hwcy}}  
\setCJKfamilyfont{hwhp}{STHupo}                                 %华文琥珀   hwhp  
\newcommand{\hwhp}{\CJKfamily{hwhp}}  
  
\setCJKfamilyfont{fzsong}{Simsun (Founder Extended)}     %方正宋体超大字符集   fzsong  
\newcommand{\fzsong}{\CJKfamily{fzsong}}  
\setCJKfamilyfont{fzyao}{FZYaoTi}                                    %方正姚体  fzy  
\newcommand{\fzyao}{\CJKfamily{fzyao}}  
\setCJKfamilyfont{fzshu}{FZShuTi}                                    %方正舒体 fzshu  
\newcommand{\fzshu}{\CJKfamily{fzshu}}  
  
\setCJKfamilyfont{asong}{Adobe Song Std}                        %Adobe 宋体  asong  
\newcommand{\asong}{\CJKfamily{asong}}  
\setCJKfamilyfont{ahei}{Adobe Heiti Std}                            %Adobe 黑体  ahei  
\newcommand{\ahei}{\CJKfamily{ahei}}  
\setCJKfamilyfont{akai}{Adobe Kaiti Std}                            %Adobe 楷体  akai  
\newcommand{\akai}{\CJKfamily{akai}}  
   
%------------------------------设置字体大小------------------------%  
%\newcommand{\chuhao}{\fontsize{42pt}{\baselineskip}\selectfont}     %初号  
%\newcommand{\xiaochuhao}{\fontsize{36pt}{\baselineskip}\selectfont} %小初号  
%\newcommand{\yihao}{\fontsize{28pt}{\baselineskip}\selectfont}      %一号  
%\newcommand{\erhao}{\fontsize{21pt}{\baselineskip}\selectfont}      %二号  
%\newcommand{\xiaoerhao}{\fontsize{18pt}{\baselineskip}\selectfont}  %小二号  
%\newcommand{\sanhao}{\fontsize{15.75pt}{\baselineskip}\selectfont}  %三号  
%\newcommand{\sihao}{\fontsize{14pt}{\baselineskip}\selectfont}%     四号  
%\newcommand{\xiaosihao}{\fontsize{12pt}{\baselineskip}\selectfont}  %小四号  
%\newcommand{\wuhao}{\fontsize{10.5pt}{\baselineskip}\selectfont}    %五号  
%\newcommand{\xiaowuhao}{\fontsize{9pt}{\baselineskip}\selectfont}   %小五号  
%\newcommand{\liuhao}{\fontsize{7.875pt}{\baselineskip}\selectfont}  %六号  
%\newcommand{\qihao}{\fontsize{5.25pt}{\baselineskip}\selectfont}    %七号  



\renewcommand{\baselinestretch}{1.5}%行间距
\newcommand{\chuhao}{\fontsize{42pt}{42pt}\selectfont}
\newcommand{\xiaochuhao}{\fontsize{36pt}{36pt}\selectfont}
\newcommand{\yihao}{\fontsize{28pt}{28pt}\selectfont}
\newcommand{\erhao}{\fontsize{21pt}{21pt}\selectfont}
\newcommand{\xiaoerhao}{\fontsize{18pt}{18pt}\selectfont}
\newcommand{\sanhao}{\fontsize{15.75pt}{15.75pt}\selectfont}
\newcommand{\sihao}{\fontsize{14pt}{14pt}\selectfont}
\newcommand{\xiaosihao}{\fontsize{12pt}{12pt}\selectfont}
\newcommand{\wuhao}{\fontsize{10.5pt}{10.5pt}\selectfont}
\newcommand{\xiaowuhao}{\fontsize{9pt}{9pt}\selectfont}
\newcommand{\liuhao}{\fontsize{7.875pt}{7.875pt}\selectfont}
\newcommand{\qihao}{\fontsize{5.25pt}{5.25pt}\selectfont}

%%%%%%%%%%%%%%%%%%%%%%%%%%%%%%%%%%%%%%%%%%%%%%%%%%%%%%%%%%

%字间距
%只适用于CJK和xeCJK
%\renewcommand{\CJKglue}{\hskip 宽度}
%比如
%\renewcommand{\CJKglue}{\hskip 1pt plus 0.08\baselineskip}
%行间距
%\linespread{因子}
%或者
%\renewcommand{\baselinestretch}{因子}
%比如
%\linespread{1.5}
%\selectfont
%或者
%\renewcommand{\baselinestretch}{1.5}
%段间距
%设置\parskip的值,比如
%\setlength{\parskip}{0.5\baselineskip}
%首行缩进
%如果默认首行不缩进,则使用indentfirst宏包
\usepackage{indentfirst}
%指定某段首行缩进,在段首加
%\indent
%指定某段首行不缩进,在段首加
%\noindent
%设置缩进量
\setlength\parindent{2em}





%%%%%%%%%%%%%%%%%%%%%%%%%%%%%%%%%%%%%%%%%%%%%%%%%%%%%%%%%%%
%%%%%%%%%%%%%%%%设置页眉页脚

\usepackage{fancyhdr} 
%\pagestyle{fancy}
%\pagestyle{empty}                   %不设置页眉页脚            
\footskip = 10pt                                                
\pagestyle{fancy}                    % 设置页眉    
\lhead{\thepage}                    
\chead{页眉中间}                                                
\rhead{\small\leftmark}                                                
\cfoot{\thepage}                                                
\rfoot{页脚左}%                                                       
\lfoot{页脚右}                                                        
\renewcommand{\headrulewidth}{1pt}  %页眉线宽,设为0可以去页眉线
\setlength{\skip\footins}{0.5cm}    %脚注与正文的距离           
\renewcommand{\footnotesize}{}      %设置脚注字体大小           
\renewcommand{\footrulewidth}{1pt}  %脚注线的宽度  

%\setlength{\footheight}{0.5cm}   

%%%%%%%%%%%%%%%% 添加底色 %%%%%%%%%%%%%%%%%%%%%

\usepackage{xcolor}
\usepackage[normalem]{ulem} % use normalem to protect \emph
\newcommand\hl{\bgroup\markoverwith
  {\textcolor{yellow}{\rule[-.5ex]{2pt}{2.5ex}}}\ULon}


%这样我们使用\hl命令即可添加高亮。这里设置的高亮色为黄色




%%%%%%%%%%%%%%%%%%%%%%%%%%%%%%%%%%%%%%%%%%%%%%%%%%%%%%%

%设置支持超链接



 \usepackage[colorlinks,linkcolor=red,anchorcolor=blue,citecolor=green]{hyperref}




%%%%%%%%%%%%%%%%%%%%%%%%%%%%%%%%%%%%%%%%%%%%%%%

\renewcommand{\contentsname}{目录}

%%%%%%%%%%%%%%%%%%%%%%%%%%%%%%%%%%%%%%%%%%%%%%%%%%

\begin{document}
\tableofcontents
\clearpage


%另起一页才能显示目录
 \hl{藤原龙也(Tatsuya Fujiwara),1982年5月15日出生于日本埼玉县秩父市,日本演员。日本戏剧史上最年轻的舞台最高奖得主,20岁出头就横扫了日本演剧界的所有大奖。舞台上的他早已被断言为“背负日本戏剧界未来三十年希望的天才役者”。他不仅在舞台剧上有着惊人的天赋与出色的表现,在电影方面也为观众带来过不少好片。}\footnotemark[1]
\footnotetext[1]{这个是个测试在开始的时候}


%\twocolumn设置双栏显示


\section{大逃杀}
为了培养出忠实效忠于成人、在逆境中坚忍不拔的青少年一代,日本政府出台《BR》法案。每年都从全国学校随机抽出一个班级的同学,前往荒岛进行生存极限挑战——老师发给学生地图、粮食和各式武器,令他们自相残杀,直到存活下来的最后一个,才能离开荒岛。接下来,残酷的游戏规则和令人绝望的生存条件,使班级里的年轻人开始了相互杀戮。善良或者凶残,主动出击或者被动防守, 同学们开始了各自的计划,人性的丑恶在血腥的死亡中暴露无遗。
大逃杀的游戏在荒岛上壮烈上演。究竟学生们的宿命如何,谁才是最后的存活者。
\subsection{大逃杀}
通杀了没?

%\onecolumn

\section{赌博默示录}
已经奔三的伊藤开司做着无聊乏味的便利店员工作,对财富的渴望换来的不过是些无奈失落。某日,一位名叫远藤澟子的女人带领手下找到伊藤,要求他作为担保人偿还巨额高利贷债务。拮据的伊藤无力清债,只得按照远藤的建议登上“希望号”赌船奋力一搏。但在上船的那一刻起,伊藤以及其他众多同病相连的所谓“失败者”便陷入了巨大财富帝国的圈套。在赌博游戏中失败的伊藤众人,从此沦为帝国的奴隶,日日在地下工程中劳作,以期有朝一日能够脱离债务重返自由。帝国主管利川根等人冷血无情,用尽招数榨取奴隶价值,不堪忍受剥削的伊藤最终选择了九死一生的“勇者之路”,用生命为赌注向利川根发起挑战……

本片根据著名同名漫画改编
赌博默示录:人生逆转游戏 上集【中字】
\footnotemark[2]
\footnotetext[2]{这是一个教学示例还可以吧}

\section{三分之一}
在俱乐部Honey Bunny担任雇佣店长的清原修造赌马时弄丢店里的营业额,为免被老板破魔翔杀死,他被俱乐部头牌茉莉亚拉去向恐怖魔女涉柿多见子举债。拆东墙补西墙的做法无济于事,修造找来最信赖自己的小弟小岛一德和生意濒临破产的常客金森健,策划了抢劫银行的行动。抢劫当天,三人事成后回到Honey Bunny,可是分赃不均产生内讧。而在这次银行劫案背后,又有多方势力以及阴谋介入其中。尔虞我诈,坑蒙拐骗,谁也不知结局将会如何……
本片根据木下半太的漫画原作改编。
三分之一
\footnotemark[3]
\footnotetext[3]{这是一个教学示例还可以吧}

\noindent
\hangafter=1
\setlength{\hangindent}{2em}

\section{稻草之盾}
谁也没有想到,一起惨无人道的女童虐杀案件竟会让全日本上下陷入疯狂。遇害者的爷爷蜷川隆兴是在经济领域执牛耳的大人物,这名悲痛至极的老人在发行量最大的报纸上刊登悬赏启示,凡能杀死嫌疑人清丸国秀的人,将得到10亿日元的酬金。这则前所未有的启示很快发生了作用,担心自身安全的清丸在福冈县投案自首,但为了将其押送回东京送检,还将跋涉一千多公里的路程。此时此刻,日本国民已经相继化身为为了金钱不惜动武杀人的阿修罗。危急关头,SP铭苅一基、白岩笃子、奥村武、神箸正贵以及关谷贤示等五名精英承担了押送的任务。交杂着人性和法律的抗争,恶魔清丸和保护他的警察踏上漫长而凶险的旅途……

本片根据木内一裕的同名原作改编。
稻草之盾SP

\section{算计:七天的死亡游戏}
年轻的无业者结城理久彦为了工作伤尽了头脑,偶然机缘,他结识了美丽的女孩须和名祥子。在对方的启发下,结城报名参加了一个时薪高达11万日元的实验。与之一同参加实验的还有祥子、岩城、安东等9人。他们被两辆白色的豪华汽车带到位于郊外的一幢房子内,在接下来的7天里,他们将在此全封闭度过。然而高回报往往伴随着高风险,在此期间,不断有人死于非命,幸存者不仅要小心保住自己的性命,还要积极追查潜在的凶手。小小的房间内,暗影重重,危机四伏……

本片根据长泽穗信的推理小说改编。
日本最新恐怖惊悚片 算计:七天的死亡游戏
\section{举例}
\begin{verbatim}
标点。
\end{verbatim}
 
汉字Chinese数学$x=y$空格


\section{对齐方式}
\begin{center}
居中文字 \\
居中文字
\end{center}
 
\noindent
\begin{minipage}{\linewidth}
\centering
居中文字 \\
居中文字
\end{minipage}
 
\begin{flushleft}
左对齐文字 \\
左对齐文字
\end{flushleft}
 
\noindent
\begin{minipage}{\linewidth}
\raggedright
左对齐文字 \\
左对齐文字
\end{minipage}
 
\begin{flushright}
右对齐文字 \\
右对齐文字
\end{flushright}
 
\noindent
\begin{minipage}{\linewidth}
\raggedleft
右对齐文字 \\
右对齐文字
\end{minipage}


\section{超链接的问题}

 在Latex中建立到参考文献链接,有利于读者快速定位到参考文献,获致参考文件的信息。

为此需要注意2点: (1)需要引入hyperref宏包; (2)不要用latex-->dvips-->ps2pdf的方式生成pdf文件。

在引入宏包的时候,可以用如下语句导入hyperref宏包:
\\usepackage[colorlinks, citecolor=red]{hyperref}
然后用pdflatex命令编译,如此生成的pdf文件中用红色的字体表明参考文件序号,点击该序号,可以链接到文章后面的该篇参考文件处。
colorlinks选项表示显示参考文献序号时,采用颜色作为标识,若不用该选项,则采用默认的方框方式显示序号。citecolor选项表示显示参考文献序号的颜色,默认为绿色。

 如何在生成的pdf文件中包含超链接呢?需要注意一下两点:

1. 使用“hyperref”这个宏包,即在latex文档的导言部分添加“\\usepackage{hyperref}”;
2. 使用“PdfLatex”对latex源文件进行编译,不要用“Latex”编译。

 

这样能确保生成的pdf文件中包含有可以用鼠标进行点击的超链接。但是这样存在一个问题,就是这些包含超链接的文本周围会出现彩色的方框,这种方框实在有碍观瞻,尤其是当出现在目录中时,大片的方框非常难看。

客服以上问题的方法是,不要使用“hyperref”宏包的默认属性,即使用如下方式引入宏包:

 usepackage[colorlinks,linkcolor=red, anchorcolor=blue, citecolor=green]{hyperref}

“colorlinks”的意思是将超链接以颜色来标识,而并非使用默认的方框来标识。
linkcolor, anchorcolor, citecolor分别表示用来标识link, anchor, cite等各种链接的颜色。
若正式的文档中不想使用彩色的标识,但又希望具有超链接的功能,则将上例中的各种颜色换成“black”即可。

如果您的pdf制作中文书签有乱码如下命令,就OK了
\\usepackage[dvipdfm,  pdfstartview=FitH,CJKbookmarks=true,bookmarksnumbered=true,bookmarksopen=true,colorlinks, pdfborder=001, linkcolor=green,  anchorcolor=green, citecolor=green]{hyperref}

 若正式的文档中不想使用彩色的标识,但又希望具有超链接的功能,则将上例中的各种颜色换成“black”即可。
 


\section{四号字体}
\wuhao {四号
“书籍是世界珍贵的财富,是世世代代和一切国家最好的继承。最古老和最优秀的书籍自然而然地、合情合理地占据着每一所房子里的书架。它们没有自己的利益与诉求,但是在它们给读者以启迪和激励的时候,读者的常识使他不会拒绝书籍。在任何一个社会中,书籍的作者者是天生的极富魅力的精英分子,对人类发挥着比帝王们更大的影响 。当目不识丁的、也许还是鄙视一切的商人,通过魄力和勤奋挣得了垂涎已久的闲暇和衣食无忧的生活,进入了财富和时尚的圈子 以后,最终不可避免地会转向那更高的然而却难以企及的知识和财富的圈子,这时他才会意识到自己文化的残缺,以及他一切财富的空虚无用;于是他不遗余力地要使子女获得知识文化,他深刻地感到自己这方面的不足,从而证明了他的明智;就这样,他成了一个家族的缔造者。”}
\section{五号字体}

\wuhao  \asong 五号
 语言和思维是一个信号,我们通过语言与思维来更好地了解世界,了解别人的所思所想,表达我们自己的意愿,语言与现世有联系。但是它们也同样很危险,如果我们将语言和思维与现世相混淆,就会为语言所控,为思维所限而无法从语言与思维所构造的虚无之中脱身。
\clearpage
\wuhao

当我还是个小孩子的时候,我对在幼儿园和小朋友们一起练习舞蹈表现出浓厚的兴趣。我也因为认真和投入被选拔代表幼儿园参加一次文艺汇演。结果,演出那天因为贪玩忘记去了,后来因为这件事在幼儿园被留了一级。之后在我印象中,对舞蹈这件事也就没有太大的热情了。

上小学的时候,经常有很多小伙伴围绕在我的身旁,听我讲那些道听途说或者瞎编的故事。于是我对讲故事产生了新的兴趣。后来因为上课的时候和同学们讲故事被老师体罚了一下,慢慢的对讲故事的兴趣也淡漠了。

上初中的时候班上开始流行看武侠小说,不管谁弄到一本小说,都会在全班男生之间轮流看。那个时候我们白天、晚上,课上、课下投入了大量的时间在武侠小说上。因此初中三年我最大的兴趣就是看武侠小说。

进入高中后,迫于沉重的学习和考试压力,基本上生活的重心都围绕着考上一个好的大学而努力。基本上也没有什么特别的爱好和兴趣。

上大学后学习任务一下子松了很多,和来自五湖四海的同学一比一下子傻了,因为感觉什么也不会,没有兴趣、也没有特长。那个时候特别羡慕那些有艺术、体育、演讲兴趣和特长的同学。慢慢地读了一些业余书,开始写一些豆腐块文章。互联网和网络文学的兴起,加速我在网络文学创造方面的投入,虽然最终也没写出什么满堂,但整个大学期间写作算是我比较大的兴趣了。

一转眼感觉什么都还没学会,大学就毕业了,靠着专业找到了第一份工作HR。因为所学专业和工作的原因,开始热衷于在QQ群、论坛搞各种人力资源方面的交流,也做了很多线下的学习会。那两年主要兴趣就是在人力资源的线上和线下交流。

机缘巧合后来的职业转换成了销售,一干就是十多年,在销售职业上巨大投入让我对营销产生了兴趣,这也是到目前最大最持久的职业兴趣吧。

不过还是发现自己除了职业兴趣,好像没有什么特殊的兴趣和爱好。我把这一次都归结于可能是没有天赋吧。没有艺术细胞、没有演讲口才、没有体育才能、没有语言学习天赋......

可是当我认真回顾以前人生各个阶段兴趣的时候,我突然有一个发现:兴趣哪儿是什么天赋决定的啊,只不过是不同人生阶段在某个或者某几个方面,你投入了时间练习,熟练掌握一种技能后变成了一种下意识的表现,又因为这种无意识的表现强化了对某个方面的感觉和兴趣。于是在其它人面前表现出来的是你在某个方面有兴趣,而你自己可能已经忘记了你曾经投入的时间和刻意的练习。而别人更会把你表现出来的他们没有的兴趣,归结于你的天赋,因为他们更少看见你曾经的投入。



培养兴趣的循环

我用这个发现来观察近几个月的几个兴趣:健走、画简笔画、读书、写作和演讲,通过观察更加验证了我的发现。

健走:我是在持续21天每天完成健走5km之后对健走产生了兴趣,又因为在微信每天分享健走的数据,获得朋友们的点赞和鼓励进一步加深了对健走的兴趣。

画简笔画:偶然开始画第一副简笔画,每天持续练习,实时对外展示自己的“作品”,画思维导图、做读书笔记、写简笔画方面的文章、在社群做简笔画主题分享......不断的练习让我开始掌握一些技能,技能的强化和应用又让我不断感觉到并强化对简笔画方面的兴趣。

读书、写作和演讲:我参加了一个社群的“拆书”(RIA标签读书法,把书里的知识拆为己用)活动后,因为每周要交一篇读书笔记,开始了每周至少读一本书并写一篇读书笔记。每周的持续输入(读书、读别人的读书笔记)和输出(读书笔记、在简书写作),及时获得别人的反馈(交流、点赞和打赏),于是我表现出了读书和写作的兴趣。由于经常把学习收获、读书心得不断地向身边的同事、朋友分享,我又对演讲产生了兴趣。

细细想来,生活的无趣不是我们缺少天赋,只是一个小小的决定、持续的投入,然后习得一种技能,当我们下意识的应用习得技能的时候,我们会产生兴趣,兴趣又会强化我们的决定和练习,进而技能更娴熟、兴趣更牢靠。从一个小决定开始吧,让我们做一个有趣的人,过有趣的生活。
只要往前走,总能遇见新的风景新的人,虽然并不一定比旧的那班人马好,但总胜在新奇可爱。

“honey,帮我买份臭豆腐,谢啦。”在听到电话那头传来的声声甜到不行的调调,陆彦堔就跟打了鸡血一般,掀开了温暖的被窝,一个鲤鱼打挺穿戴好了一切,冲进了冰天雪地的世界。

打电话给他的不是别人,正是和他走的最近的异性朋友付雅琴,是个职业作家,长的那叫一个貌美如花,声音也是甜到要命,于是乎,他陆彦堔算是彻底贯穿了那句“窈窕淑女,君子好逑”的话,对付雅琴要他帮忙的事上,从没说过一句不字。

“靠,这风吹的,完全就是容嬷嬷对小燕子,呼呼的往死里乎啊!”站在已经排成长龙的臭豆腐摊前,陆彦堔感叹到。

付雅琴这个女人啊,包包买的是lv,护肤品用的是兰蔻,身上穿的哪有一件是低过四位数的,偏偏在对臭豆腐这件事上,算是彻底让陆彦堔跌破眼镜了。

这种闻起来臭,长的黑乎乎的玩意,吃起来估计也好不到哪里的东西,怎么就会成为那个女人的心头好呢?当陆彦堔终于在凛冽的寒风中,拿到了那份付雅琴要吃的臭豆腐后,心头不免疑惑起来。

也罢,只要女神喜欢就好,想那些没用的干啥。想到即将看到付雅琴那张不食人间烟火的小脸,男儿本色暴露无遗的陆彦堔屁颠屁颠的冲他心仪的姑娘家跑去。

“我说,你怎么还好这口?这种路边摊既不卫生也不可靠,还是油炸食品,吃了对身体也不好。”看着在一旁吃的津津有味的付雅琴,想想自己宝贵的早上睡眠时间都贡献给了买臭豆腐这件事上,一股淡淡的忧伤在他心头蔓延。

“honey,来。尝一个,吃完后保准你也爱吃。”眼前的姑娘笑得一脸灿烂,露出两颗好看的小虎牙。

免为其难的陆彦堔看着眉眼如画的付雅琴主动喂他,自己总不能为了区区一块臭豆腐而打退堂鼓吧,转而豁出去的他屏住了呼吸,张口吃下付雅琴夹到他嘴边的那块豆腐。

吃人豆腐的下场就是,无福消受这种美食的陆彦堔,感觉自己的肠胃一阵的翻江倒海。下一秒,他就飞也似的冲进了付雅琴家的卫生间,大吐特吐起来。

感觉连自己的胆汁都要吐出来的陆彦堔,以自己不忍直视的狼狈模样,彻彻底底的将臭豆腐列为自己最讨厌食物,没有之一。

在付雅琴的眼中,陆彦堔就是一天生为女人四处奔走的伟大“男同胞,”但凡是她家的浴室叫嚣着罢工而肆意冒水的时候,她大脑里面窜出来的第一个好友名字,绝对非他莫属。她总是一手捂着浴袍,一边低落着情绪给他打电话,语气中说不出的可怜兮兮:“陆彦堔,你再不来我就要沦落成花果山水帘洞里面的孙猴子了。”

每次陆彦堔大汗淋漓地帮她弄好了之后,还要在苦口婆心的开导她,不要跟一个淋浴头一般见识,只要有他这样酷的不能再酷的爷们在,哪怕她家的马桶堵住了,她也能够分分钟摆平。有些忧伤的付雅琴不免忧伤的晒月亮,这个浴室真是让她受够了。每当这时,陆彦堔都不忘打趣她:“你可别小瞧这浴室,要是没了这玩意,世界上得少了多少美女出浴图啊!”付雅琴皱着眉头回了一句:“你真猥琐!”

陆彦堔哈哈一笑,露出一排整齐的牙齿,反驳道:“身为一个男人,面对这么一位楚楚可人的美女,若是真没有什么猥琐的念头,那才不正常呐!有个词语怎么形容来着,男儿本“色”嘛……”

付雅琴穿着她那件粉嫩到不行的露肩浴袍,自顾自的照着镜子,压根就没有听进去陆彦堔的后半段话。等他说完了,付雅琴微一扭头,瞅了他一眼,故意惊讶道:“呀,你还没走呢?你看我都忘了,要给小费来着。”

“得了吧,小妞,你瞧你这左照右照,打扮的跟个狐媚似的,又要约会去啊?啥时候考虑我这号酷的不能再酷的爷们啊?”

“就你啊,我看是直接忽略的好。有大把大把的女生跟着你屁股后面打转,我呢,也有不少爱慕者屁颠屁颠的跟在身后,既然各自都那么受欢迎,何必要绑在一起彼此束缚嘛。”

说完,“啪”的一声关上了门,窸窸窣窣地换起了衣服。

陆彦堔之所以认识付雅琴这种女人并对其印象深刻,完全是源于第一次的陌生电话聊天。彼时他还在保险公司为自己的业绩而刷的不辞辛劳的时刻,他阴差阳错的看到刚去推销人员拿来的一张报表,而她付雅琴的名字赫然出现在了显眼的第一位。既然是潜在客户,火力全开的陆彦堔随手拨打了这个号码。

很多人都担心推销人员要求自己写下电话号码 而打骚扰电话,所以多半都会留下假的联系方式。很显然并没报多大希望的陆彦堔也只是抱着试试看的心态拨通了她的号码。这也是他和她的第一次对话。

“你好,请问你是付女士嘛?你现在有空听我给你介绍一下我们保险公司最近推出的这份……”

“稍等一下,我有点事。”

“好的,大概要等多久。”

“大概……快了,马上就要交配完了。”

“交配?”陆彦堔觉得这两个字所包含的信息量太过庞大,竟然他一时大脑反应迟钝,忍不住面红心跳起来。

“是呀。哦,别误会,不是我,而是我的宠物,我在帮我的金丝雀……”

“我现在是不是妨碍你交配?”他的大脑不受控制的问了这么一句不入流的话。

“还好还好。”

“应该很有意思吧,我能请你吃饭嘛,顺便你跟我讲讲你是怎么做到的……”

也就是那次奇葩的电话交流,陆彦堔头一次发现,这个女人和别人不一样,至少神奇到不行,否则也不会再他这样的花花公子心中占有一席之地了。

后来的事情,就跟很多烂俗的电视剧一般,既然能聊的来,就彼此留下了联系方式,之后各回各家,各找各妈。再后来,就有朋友说他眼光独到,看女人的方式果然与众不同。能遇到像付雅琴这样既有姿色又有钞票的女人,如果成功的勾搭上的话,那可真是能够少奋斗至少十年的大好光阴。

倒插门这事陆彦堔觉得还是算了,不过,对于勾搭上她这件事,他还是满蛮愿意为此上刀山下火海的。

陆彦堔业务拓展成了每天茶余饭后都给付雅琴发条短信,付雅琴回他:“你要追我?”

陆彦堔盯着屏幕看了好一会儿,确定以及肯定自己没有看错后,毅然回复:“如果我不追你,爱情就会将我强暴。你要是默认了,我就考虑从了你好了。”

后来付雅琴回忆这事,笑的即是好看:“那晚真是感谢你编的那几句瞎话,我才算是从电脑前的恐怖片里回到了现实,否则的话,那天晚上肯定会做噩梦的啦。而且,我想了想,我还是很期待看到你这样的大老爷们被爱情强暴的模样。”

陆彦堔自知自己被耍了,不过还是各种绅士地说:“要是有一天,天真的塌下来了,你碰到了问题,随时call me!”

付雅琴哦了一声随口问道:“你叫什么?”

“陆彦堔。”至此,他开始沦落成了付雅琴的御用跑腿工。

付雅琴有一个惊为天人的男朋友,只消一眼就已经让陆彦堔这个被公认的帅哥而无地自容。他才豁然明白浓眉大眼、一表人才这样限量版的词汇果然是会为了某些少数人的存在而量身定做。付雅琴总是会无比骄傲的在他的面前有一碴没一碴的夸耀她那星星般的男人,她说她从十五年一眼看中了他,整整相爱了十余载。陆彦堔就问,既然你们已经相识了十余年的时光,为何不坚持一起走下去呢?这可是十年的爱情马拉松长跑呐。他已经另娶她人了。付雅琴淡淡地回复,语气平静的仿佛在叙述别人的故事一般,没有半点的遗憾和难过。

陆彦堔几乎是下意识的握住了她的手,充满安慰的说:“这年头,能够不离不弃的男人都快绝迹了。每个人的初恋都是用来祭奠和怀念的,要是都能够从一而终的话,这世界上就不会有单身狗这样高大上的玩意了。‘

”哈哈,也对,要不然现在也不会有单身联盟这样的神秘组织了。“

”没错没错,单身的人都是一支隐藏的潜力股……“

”这么说来,我也赶上潮流,自豪的单着了。单身女人最好运。“付雅琴莞尔,拿出她的IPAD在上面迅速的敲击着。

”等会,你莫不是又想到什么好点子,要把我们的谈话内容曝光到你的下一个故事里吧。“

”这个点子棒棒哒。“付雅琴合上平板。”我发现呢,每次和你聊天,大脑里面总是会闪现各种有意思的Idea,而且吧,发出去后,都会迅速飚红。“

女人心,海底针。虽然阅女无数,陆彦堔还是摸不透眼前这个古灵精怪的女人到底几分真情,几分伪装。

”我肚子饿了。陪我一起出去吃臭豆腐还不好?’

付雅琴最爱的食物始终是路边其貌不扬的臭豆腐,陆彦堔算是把自打从娘胎里面残存的那点脑细胞用光了,也没有想到那种闻着臭吃着也有点怪怪味道的臭豆腐,有啥能够值得这样的女人情有独钟的。

我刚来北京的时候,特别怀念家乡的臭豆腐,找遍了整个北京,也就这个摊位的味道和我记忆深处的味道最为相似。付雅琴用竹签插了一块炸的金黄的臭豆腐放至嘴边,继续叙说,"后来,每次我有新书签售的时候,我都会去全国各地寻找臭豆腐,比如罗家臭豆腐、火宫殿臭豆腐、黑色灌汁臭豆腐。每次吃的时候,心里都高兴到不行,可是每次吃完后,就再也不想去吃了。"

陆彦堔看着付雅琴,看着她吃完最后的那块臭豆腐,优雅的擦着嘴角的模样。

“我爱的人在我二十二岁毕业的那天离开了我,那天我什么挽留的话都没有对他说,偏偏只是要求他为我做最后一件事,就是为我做一份他最爱的一碗臭豆腐。有意思的是,他那么喜欢并且研究了好久的一份食物,他那天竟然破天荒的做的手忙脚乱,最后竟然全部统统炸的焦糊焦糊,还一个劲的和我说什么对不起。你说,我和他在一起那么多年,付出了那么多,最后就是为了听他说几声对不起嘛?我不在乎全天下的人怎么看我,可我真的很在意,我最爱的那个人,最后留给的却是我最不想听到的那句对不起。”

付雅琴仰着75度的小脸仰望着天际:“从那以后,我们就成了陌路人。说真心话,他的确是个不错的男朋友,至少,他能做出全天下最好的臭豆腐,就冲这点,我就很崇拜他。”

陆彦堔的一口汽水哽在喉头,呛得他差点以为自己要去阎王爷那里报到去了,到底是哪个人说的,要想掌控一个男人要先掌控他的胃来着。就这一刻,他突然无比的坚信,若要想泡到一个气质非凡的姑娘,必然需要抢先搞定她的吃货本性。

付雅琴破天荒的主动约 陆彦堔出门,电话里面柔声细语的音调真让陆彦堔惊喜的以为天外在下红雨,干脆请了一天的假。付雅琴约他出门,完全是出于没找到合适的人陪她逛街

,而她逛街的目的自然是为了买件好看的衣服,而买衣服的目的却是为了参加吃货厨艺秀。逛了整整一下午,始终在为了哪件衣服好看儿犹豫不决的付雅琴,突然拽着陆彦堔的一角,满脸羞涩的说:“我有选择纠结症,你帮我拿个主意,到底要买哪个。”最终还是陆彦堔强烈建议她买那件白色蕾丝的及膝长裙,毕竟从他的出发点来看,只有白衣翻飞的女人才能更让他心动不已,尤其是付雅琴这样让他产生牡丹花下死做鬼也风流的女神。

“你怎么那么喜欢臭豆腐,竟然还要靠这个参加比赛?”陆彦堔像看外星人的眼光打量着她。

“是啊,这很奇葩么?”付雅琴扑闪着她那双大大的眼睛回望着他。

这何止是奇葩啊,简直就是奇葩中的战斗机啊,这要是不开个外挂神马的,估计海选就要OUT了。不过这么打击人的话,他陆彦堔是打死了也不会当着他女神的面说明的。“你觉得你一块块黑乎乎的豆腐能让你夺魁?”

“你不相信我的手艺?”

"哪能啊!不信你不就等于不相信中国共产党吗。哦,对了,你是不是有什么秘密配方,比如祖传啥的?"

“有秘方的掺杂的臭豆腐还能称作臭豆腐嘛。就像爱情,如果夹杂了太多的念头顾虑,没有最单纯的心动和情愫,那还能称为爱情吗?”

事实上,付雅琴比赛那天,虽然公务缠身,表面上对于她的邀请模棱两可,陆彦堔还是冒着请假一天,扣双倍工资的大不违偷偷来到了她的比赛现场,而且还藏好了位置,防止她一眼就在人群中看到了他。

付雅琴的手艺极好,在第一轮海选上,让吃过的人都赞叹不已。唯独只有一个评委,将她的作品评价的一无是处,说她竟然将这样难登大雅之堂的东西搬到了正规的比赛上。

在台下越听越气不过的陆彦堔二话没说就冲上了擂台,想都没想就将那个评委老师面前的桌子掀翻了去。

“他是我宠的女人,我都舍不得欺负她。你他娘的凭什么说她做的不好,有本事你做个更好的?就你这破比赛,我还不稀罕让她来参加呢!”

说罢,当着全场观众错愕的神情下,主动拉着付雅琴的手,将她带出了现场。

“那个人就是当年的他,你说他是有多么不想见到我,还是不想回忆起当年的那个他呢?”付雅琴在事后对陆彦堔问出了这句话。

那天,陆彦堔陪着付雅琴聊了好久好久。他不知道,那次促膝长谈,竟是他看她的最后一眼,没多久,街角的那家臭豆腐摊搬走了一起离开的还有付雅琴。

同一天,陆彦堔几乎动用了他所有的关系,向他周围认识付雅琴的朋友问了这个一个问题:“你说……他曾经有没有和一个厨艺很好的男人交往,比如做臭豆腐的?”

所有的几乎都是以看待怪物的神情诧异的回应:“不可能吧,像她这种写写字出出书就能赚好多钱的文艺女神范,再怎么着也不会眼光这么独到。”

一年后,由于工作的调动,陆彦堔来到了上海,正巧住宿的地方有个小吃一条街,路口刚好就有家专门卖臭豆腐的。他几乎是鬼使神差的走了过去,跟老板要了一份臭豆腐恍惚间,他好像又回到了那年夏天,笑容灿烂的付雅琴正在街边的一张桌子上吃着臭豆腐,也就是在那个特殊场景下,他终是下定决心问出了心中的那句话:“你就不能试着喜欢我一下嘛,就一下也好。”

“我有说过我不喜欢你嘛。”她淡淡说着。

“但可是……”

他还在迟疑的片刻,她忽然起身而又俯下身子,在他的面颊上印上了一个极浅的吻 ,如蜻蜓点水。

“这就是你所谓的喜欢吗?”

“是……也不全是,还差点……”

“如果我说,在吻你之前我是真的很喜欢你,可是吻你之后的下一秒,我却没那么喜欢你了,你会怎样?”

“这是神马逻辑?”

“你们男人所谓的喜欢,不都很喜欢用行动来证明。比如接吻,比如拥抱,比如做爱,也不过就是想要私自占有,预防别人抢走而已。难道默默的守候和陪伴在一个人的左右,不是最长情的告白?”

陆彦堔被说的竟一时不知该如何辩解。

”陆彦堔,直着往前走在向左拐就是民政局了,我们俩一起去拍个合照领个红本本吧。“

”啊,现在……“事情转变的太突然,陆彦堔只觉得自己的大脑已经开始停止运转了。

”你看,这世界上最能证明一段感情的最好方式就是去领证,而最不靠谱的也是这个小本本,它可以让两个素昧平生十多年不曾出现在彼此生命中的两个人紧紧绑在一起,却又不能证明他们才是这个世界上的真爱,这样的喜欢,无非就是一个寂寞的人想要找个床伴罢了。“

“雅倩,你知道,我不是……”

‘那你想要得到我的什么呢?“

这句话刚一说完,她的手机就响了,也不知道是她的现任第几个男朋友打给她的,说约她晚上一起逛夜市,当她挂完手机后,微微一笑,说道,你看,世界上这么多男人,都等着我来回应他们所谓的喜欢,你也想成为其中之一么?

十一的时候,陆彦堔心血来潮的从网上拍了份臭豆腐,在淘宝掌柜的不懈指导下,有模有样的炸起了臭豆腐。当他那碗香气四溢的臭豆腐端上桌的时候。手机突然传来了一阵铃声,竟然是付雅琴的来电,她说:”忙到现在,饿死我了,晚上陪我吃饭吧?“

陆彦堔会心一笑,回了句:”好啊,你来我家吧,一起吃臭豆腐可好?“
\end{document}





