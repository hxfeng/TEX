% !TEX program = xelatex
% !Mode:: "TeX:UTF-8"
\documentclass[12pt,a4paper]{article}
\usepackage[top=25.4mm,bottom=25.4mm,left=31.7mm,right=31.7mm]{geometry} %设置页边距
\usepackage[no-math]{fontspec} %提供字体选择命令
\usepackage{xunicode}  %提供 Unicode 字符宏
\usepackage{xltxtra}   %提供了一些针对XeTeX的改进并且加入了XeTeX的LOGO
\usepackage[slantfont,boldfont]{xeCJK} %使用xeCJK宏包
\usepackage{amsmath}
\usepackage{graphics}%图形宏包
\usepackage{xcolor}  %颜色宏包
\usepackage{titleps}  %设置页眉,页脚
\usepackage{lettrine} %设置首字下沉
\usepackage[dvipsnames]{pstricks}
\usepackage{pst-plot}   %ps宏包
\usepackage{pst-func}    %画函数的宏包
%%-----------------------设置中文字体-----------------------------------------------------------------------------------------
\setCJKmainfont[BoldFont=Adobe Heiti Std,ItalicFont=Adobe Kaiti Std]{Adobe Song Std} %设置正文为宋体,粗体使用黑体,斜体使用楷体
\setCJKmonofont{Adobe Song Std} %设置等距字体
\setCJKsansfont[BoldFont=Adobe Heiti Std]{Adobe Kaiti Std} %设置无衬线字体
%%-----------------------设置英文字体-----------------------------------------------------------------------------------------
\setmainfont[Mapping=tex-text]{TeX Gyre Pagella} %英文衬线字体
\setsansfont[Mapping=tex-text]{Trebuchet MS}     %英文无衬线字体
\setmonofont[Mapping=tex-text]{Courier New}      %英文等宽字体
%%%%----------重定义中文字体%%%%---------------
\setCJKfamilyfont{song}{Adobe Song Std}
\setCJKfamilyfont{hei}{Adobe Heiti Std}
\setCJKfamilyfont{kai}{Adobe Kaiti Std}
\setCJKfamilyfont{fs}{Adobe Fangsong Std}
%%-------------------------------------
\newcommand{\song}{\CJKfamily{song}}
\newcommand{\hei}{\CJKfamily{hei}}
\newcommand{\kai}{\CJKfamily{kai}}
\newcommand{\fs}{\CJKfamily{fs}}
%%----------------字号命令--------------------------------------------
\newcommand{\chuhao}{\fontsize{42.2pt}{\baselineskip}\selectfont}
\newcommand{\xiaochuhao}{\fontsize{36.1pt}{\baselineskip}\selectfont}
\newcommand{\yihao}{\fontsize{26.1pt}{\baselineskip}\selectfont}
\newcommand{\xiaoyihao}{\fontsize{24.1pt}{\baselineskip}\selectfont}
\newcommand{\erhao}{\fontsize{22.1pt}{\baselineskip}\selectfont}
\newcommand{\xiaoerhao}{\fontsize{18.1pt}{\baselineskip}\selectfont}
\newcommand{\sanhao}{\fontsize{16.1pt}{\baselineskip}\selectfont}
\newcommand{\xiaosanhao}{\fontsize{15.1pt}{\baselineskip}\selectfont}
\newcommand{\sihao}{\fontsize{14.1pt}{\baselineskip}\selectfont}
\newcommand{\xiaosihao}{\fontsize{12.1pt}{\baselineskip}\selectfont}
\newcommand{\wuhao}{\fontsize{10.5pt}{\baselineskip}\selectfont}
\newcommand{\xiaowuhao}{\fontsize{9.0pt}{\baselineskip}\selectfont}
\newcommand{\liuhao}{\fontsize{7.5pt}{\baselineskip}\selectfont}
\newcommand{\xiaoliuhao}{\fontsize{6.5pt}{\baselineskip}\selectfont}
\newcommand{\qihao}{\fontsize{5.5pt}{\baselineskip}\selectfont}
\newcommand{\bahao}{\fontsize{5.0pt}{\baselineskip}\selectfont}
%%----------------------------------------------------------------
\punctstyle{kaiming} %开明式标点格式
\linespread{1.5}   % 1.5倍行距
\begin{document}
\newpagestyle{yang}{
\sethead{LanZhou University of Technology}{}{兰州理工大学}
\setfoot{数学系}{}{第~~\thepage ~~页}\headrule\footrule}
\pagestyle{yang}
%-------------------------------------------------------
{\color{red}\chuhao 二十四孝}

\title{\color{blue} 01 孝感动天}

\lettrine[lhang=1, nindent=0pt,lines=2]{舜}{}传说中的远古帝王,五帝之一,姓姚,名重华,号有虞氏,史称虞舜。相传他的父亲瞽叟及继母、
异母弟象,多次想害死他:让舜修补谷仓仓顶时,从谷仓下纵火,舜手持两个斗笠跳下逃脱;让舜掘井时,瞽叟与象却下土填井,舜掘地道逃脱。事后舜毫不嫉恨,仍对父亲恭顺,对弟弟慈爱。他的孝行感动了天帝。舜在厉山耕种,大象替他耕地,鸟代他锄草。帝尧听说舜非常孝顺,有处理政事的才干,把两个女儿娥皇和女英嫁给他;经过多年观察和考验,选定舜做他的继承人。舜登天子位后,去看望父亲,仍然恭恭敬敬,并封象为诸侯。

02 亲尝汤药

\lettrine[lines=2]{汉}{}文帝刘恒,汉高祖第三子,为薄太后所生。高后八年(前180)即帝位。他以仁孝之名,闻于天下,侍奉母亲从不懈怠。母亲卧病三年,他常常目不交睫,衣不解带;母亲所服的汤药,他亲口尝过后才放心让母亲服用。他在位24年,重德治,兴礼仪,注意发展农业,使西汉社会稳定,人丁兴旺,经济得到恢复和发展,他与汉景帝的统治时期被誉为“文景之治”。

03 啮指痛心

\lettrine[lines=1]{曾}{}参,字子舆,春秋时期鲁国人,孔子的得意弟子,世称“曾子”,以孝著称。少年时家贫,常入山打柴。一天,家里来了客人,母亲不知所措,就用牙咬自己的手指。曾参忽然觉得心疼,知道母亲在呼唤自己,便背着柴迅速返回家中,跪问缘故。母亲说:“有客人忽然到来,我咬手指盼你回来。”曾参于是接见客人,以礼相待。曾参学识渊博,曾提出“吾日三省吾身”(《论语·学而》)的修养方法,相传他著述有《大学》、《孝经》等儒家经典,后世儒家尊他为“宗圣”。

04 百里负米

仲由,字子路、季路,春秋时期鲁国人,孔子的得意弟子,性格直率勇敢,十分孝顺。早年家中贫穷,自己常常采野菜做饭食,却从百里之外负米回家侍奉双亲。父母死后,他做了大官,奉命到楚国去,随从的车马有百乘之众,所积的粮食有万钟之多。坐在垒叠的锦褥上,吃着丰盛的筵席,他常常怀念双亲,慨叹说:“即使我想吃野菜,为父母亲去负米,哪里能够再得呢?”孔子赞扬说:“你侍奉父母,可以说是生时尽力,死后思念哪!”(《孔子家语·致思》)

05 芦衣顺母

闵损,字子骞,春秋时期鲁国人,孔子的弟子,在孔门中以德行与颜渊并称。孔子曾赞扬他说:“孝哉,闵子骞!”(《论语·先进》)。他生母早死,父亲娶了后妻,又生了两个儿子。继母经常虐待他,冬天,两个弟弟穿着用棉花做的冬衣,却给他穿用芦花做的“棉衣”。一天,父亲出门,闵损牵车时因寒冷打颤,将绳子掉落地上,遭到父亲的斥责和鞭打,芦花随着打破的衣缝飞了出来,父亲方知闵损受到虐待。父亲返回家,要休逐后妻。闵损跪求父亲饶恕继母,说:“留下母亲只是我一个人受冷,休了母亲三个孩子都要挨冻。”父亲十分感动,就依了他。继母听说,悔恨知错,从此对待他如亲子。

06 鹿乳奉亲

郯子,春秋时期人。父母年老,患眼疾,需饮鹿乳疗治。他便披鹿皮进入深山,钻进鹿群中,挤取鹿乳,供奉双亲。一次取乳时,看见猎人正要射杀一只麂鹿,郯子急忙掀起鹿皮现身走出,将挤取鹿乳为双亲医病的实情告知猎人,猎人敬他孝顺,以鹿乳相赠,护送他出山。

07 戏彩娱亲

老莱子,春秋时期楚国隐士,为躲避世乱,自耕于蒙山南麓。他孝顺父母,尽拣美味供奉双亲,70岁尚不言老,常穿着五色彩衣,手持拨浪鼓如小孩子般戏耍,以博父母开怀。一次为双亲送水,进屋时跌了一跤,他怕父母伤心,索性躺在地上学小孩子哭,二老大笑。

08 卖身葬父

董永,相传为东汉时期千乘(今山东高青县北)人,少年丧母,因避兵乱迁居安陆(今属湖北)。其后父亲亡故,董永卖身至一富家为奴,换取丧葬费用。上工路上,于槐荫下遇一女子,自言无家可归,二人结为夫妇。女子以一月时间织成三百匹锦缎,为董永抵债赎身,返家途中,行至槐荫,女子告诉董永:自己是天帝之女,奉命帮助董永还债。言毕凌空而去。因此,槐荫改名为孝感。

09 刻木事亲

丁兰,相传为东汉时期河内(今河南黄河北)人,幼年父母双亡,他经常思念父母的养育之恩,于是用木头刻成双亲的雕像,事之如生,凡事均和木像商议,每日三餐敬过双亲后自己方才食用,出门前一定禀告,回家后一定面见,从不懈怠。久之,其妻对木像便不太恭敬了,竟好奇地用针刺木像的手指,而木像的手指居然有血流出。丁兰回家见木像眼中垂泪,问知实情,遂将妻子休弃。

10 行佣供母

江革,东汉时齐国临淄人,少年丧父,侍奉母亲极为孝顺。战乱中,江革背着母亲逃难,几次遇到匪盗,贼人欲杀死他,江革哭告:老母年迈,无人奉养,贼人见他孝顺,不忍杀他。后来,他迁居江苏下邳,做雇工供养母亲,自己贫穷赤脚,而母亲所需甚丰。明帝时被推举为孝廉,章帝时被推举为贤良方正,任五官中郎将。

11 怀橘遗亲

陆绩,三国时期吴国吴县华亭(今上海市松江)人,科学家。六岁时,随父亲陆康到九江谒见袁术,袁术拿出橘子招待,陆绩往怀里藏了两个橘子。临行时,橘子滚落地上,袁术嘲笑道:“陆郎来我家作客,走的时候还要怀藏主人的橘子吗?”陆绩回答说:“母亲喜欢吃橘子,我想拿回去送给母亲尝尝。”袁术见他小小年纪就懂得孝顺母亲,十分惊奇。陆绩成年后,博学多识,通晓天文、历算,曾作《浑天图》,注《易经》,撰写《太玄经注》。

12 埋儿奉母

郭巨,晋代隆虑(今河南林县)人,一说河内温县(今河南温县西南)人,原本家道殷实。父亲死后,他把家产分作两份,给了两个弟弟,自己独取母亲供养,对母极孝。后家境逐渐贫困,妻子生一男孩,郭巨担心,养这个孩子,必然影响供养母亲,遂和妻子商议:“儿子可以再有,母亲死了不能复活,不如埋掉儿子,节省些粮食供养母亲。”当他们挖坑时,在地下二尺处忽见一坛黄金,上书“天赐郭巨,官不得取,民不得夺”。夫妻得到黄金,回家孝敬母亲,并得以兼养孩子。

13 扇枕温衾

黄香,东汉江夏安陆人,九岁丧母,事父极孝。酷夏时为父亲扇凉枕席;寒冬时用身体为父亲温暖被褥。少年时即博通经典,文采飞扬,京师广泛流传“天下无双,江夏黄童”。安帝(107-125年)时任魏郡(今属河北)太守,魏郡遭受水灾,黄香尽其所有赈济灾民。著有《九宫赋》、《天子冠颂》等。

14 拾葚异器

蔡顺,汉代汝南(今属河南)人,少年丧父,事母甚孝。当时正值王莽之乱,又遇饥荒,柴米昂贵,只得拾桑葚母子充饥。一天,巧遇赤眉军,义军士兵厉声问道:“为什么把红色的桑葚和黑色的桑葚分开装在两个篓子里?”蔡顺回答说:“黑色的桑葚供老母食用,红色的桑葚留给自己吃。” 赤眉军怜悯他的孝心,送给他三斗白米,一头牛,带回去供奉他的母亲,以示敬意。

15 涌泉跃鲤

姜诗,东汉四川广汉人,娶庞氏为妻。夫妻孝顺,其家距长江六七里之遥,庞氏常到江边取婆婆喜喝的长江水。婆婆爱吃鱼,夫妻就常做鱼给她吃,婆婆不愿意独自吃,他们又请来邻居老婆婆一起吃。一次因风大,庞氏取水晚归,姜诗怀疑她怠慢母亲,将她逐出家门。庞氏寄居在邻居家中,昼夜辛勤纺纱织布,将积蓄所得托邻居送回家中孝敬婆婆。其后,婆婆知道了庞氏被逐之事,令姜诗将其请回。庞氏回家这天,院中忽然喷涌出泉水,口味与长江水相同,每天还有两条鲤鱼跃出。从此,庞氏便用这些供奉婆婆,不必远走江边了。

16 闻雷泣墓

王裒,魏晋时期营陵(今山东昌乐东南)人,博学多能。父亲王仪被司马昭杀害,他隐居以教书为业,终身不面向西坐,表示永不作晋臣。其母在世时怕雷,死后埋葬在山林中。每当风雨天气,听到雷声,他就跑到母亲坟前,跪拜安慰母亲说:“裒儿在这里,母亲不要害怕。”他教书时,每当读到《蓼莪》篇,就常常泪流满面,思念父母。

17 乳姑不怠

崔山南,名,唐代博陵(今属河北)人,官至山南西道节度使,人称“山南”。当年,崔山南的曾祖母长孙夫人,年事已高,牙齿脱落,祖母唐夫人十分孝顺,每天盥洗后,都上堂用自己的乳汁喂养婆婆,如此数年,长孙夫人不再吃其他饭食,身体依然健康。长孙夫人病重时,将全家大小召集在一起,说:“我无以报答新妇之恩,但愿新妇的子孙媳妇也像她孝敬我一样孝敬她。”后来崔山南做了高官,果然像长孙夫人所嘱,孝敬祖母唐夫人。

18 卧冰求鲤

王祥,琅琊人,生母早丧,继母朱氏多次在他父亲面前说他的坏话,使他失去父爱。父母患病,他衣不解带侍候,继母想吃活鲤鱼,适值天寒地冻,他解开衣服卧在冰上,冰忽然自行融化,跃出两条鲤鱼。继母食后,果然病愈。王祥隐居二十余年,后从温县县令做到大司农、司空、太尉。

19 恣蚊饱血

吴猛,晋朝濮阳人,八岁时就懂得孝敬父母。家里贫穷,没有蚊帐,蚊虫叮咬使父亲不能安睡。每到夏夜,吴猛总是赤身坐在父亲床前,任蚊虫叮咬而不驱赶,担心蚊虫离开自己去叮咬父亲。

20 扼虎救父

杨香,晋朝人。十四岁时随父亲到田间割稻,忽然跑来一只猛虎,把父亲扑倒叼走,杨香手无寸铁,为救父亲,全然不顾自己的安危,急忙跳上前,用尽全身气力扼住猛虎的咽喉。猛虎终于放下父亲跑掉了。

21 哭竹生笋

孟宗,三国时江夏人,少年时父亡,母亲年老病重,医生嘱用鲜竹笋做汤。适值严冬,没有鲜笋,孟宗无计可施,独自一人跑到竹林里,扶竹哭泣。少顷,他忽然听到地裂声,只见地上长出数茎嫩笋。孟宗大喜,采回做汤,母亲喝了后果然病愈。后来他官至司空。

22 尝粪忧心

庾黔娄,南齐高士,任孱陵县令。赴任不满十天,忽觉心惊流汗,预感家中有事,当即辞官返乡。回到家中,知父亲已病重两日。医生嘱咐说:“要知道病情吉凶,只要尝一尝病人粪便的味道,味苦就好。” 黔娄于是就去尝父亲的粪便,发现味甜,内心十分忧虑,夜里跪拜北斗星,乞求以身代父去死。几天后父亲死去,黔娄安葬了父亲,并守制三年。

23 弃官寻母

朱寿昌,宋代天长人,七岁时,生母刘氏被嫡母(父亲的正妻)嫉妒,不得不改嫁他人,五十年母子音信不通。神宗时,朱寿昌在朝做官,曾经刺血书写《金刚经》,行四方寻找生母,得到线索后,决心弃官到陕西寻找生母,发誓不见母亲永不返回。终于在陕州遇到生母和两个弟弟,母子欢聚,一起返回,这时母亲已经七十多岁了。

24 涤亲溺器

黄庭坚,北宋分宁(今江西修水)人,著名诗人、书法家。虽身居高位,侍奉母亲却竭尽孝诚,每天晚上,都亲自为母亲洗涤溺器(便桶),没有一天忘记儿子应尽的职责。

\parbox[c]{100pt}{锦瑟无端五十弦,\\一弦一柱思华年。}李商隐
%---------------------------------------------------------
\begin{center}
\begin{pspicture*}(-3,-3.2)(3.5,3.5)
\psaxes[ticks=none,labels=none]{->}(0,0)(-3,-3)(3.2,3)%
\psplotImp[linewidth=2pt,linecolor=green](-6,-6)(4,2.4){%
 x 3 exp y 3 exp add 4 x y mul mul sub }
\psplot[linecolor=red,linestyle=dashed]{-6}{4}{x neg 1.333 sub}
\end{pspicture*}
\end{center}
%%%------------------------------
\end{document} 