% !Mode:: "TeX:UTF-8"
% !TEX program = xelatex
\documentclass[CJKnumber]{ctexart}
\usepackage[a4paper,top=3cm,bottom=2cm,left=3cm,right=3cm]{geometry}
\usepackage{fancyhdr}
\fancypagestyle{verticle}{
  \renewcommand\thepage{\CJKdigits{\arabic{page}}}
  \renewcommand{\headrulewidth}{0pt}
  \fancyhf{}
  \rhead{\normalfont
    \unitlength=1cm
    \begin{picture}(0,0)
      \put (1,-1) {\rotatebox{-90}{第\thepage{}頁}}
    \end{picture}}
}

\CTEXsetup[name={,}, number=\chinese{section}, format=\large\bfseries]{section}

\newfontlanguage{Chinese}{CHN}
\setCJKfamilyfont{songvert}[Script=CJK,Language=Chinese,Vertical=RotatedGlyphs]{SimSun}
\newcommand*\songvert{\CJKfamily{songvert}
\punctstyle{quanjiao}\CJKmove}

\newcommand*\CJKmovesymbol[1]{\raise.3em\hbox{#1}}
\newcommand*\CJKmove{%
  \let\CJKsymbol\CJKmovesymbol
  \let\CJKpunctsymbol\CJKsymbol}


\usepackage{lscape}

\makeatletter
\let\antilandscape\landscape
\let\endantilandscape\endlandscape
\def\LS@antirot{%
  \setbox\@outputbox\vbox{\hbox{\rotatebox{-90}{\box\@outputbox}}}}
\patchcmd{\antilandscape}{\LS@rot}{\LS@antirot}{}{}

\newenvironment{CJKvert}
  {\antilandscape
   \appto{\normalfont}{\songvert}\songvert
   \let\oldfootnote\footnote
   \renewcommand\footnote[1]{\oldfootnote{\songvert##1}}%
   \renewcommand\thefootnote{[\CJKdigits{\arabic{footnote}}]}%
   \def\nosup@makefnmark{\hbox{\normalfont\@thefnmark\space}}%
   \let\old@makefntext\@makefntext
   \long\def\@makefntext##1{{%
     \let\@makefnmark\nosup@makefnmark
     \old@makefntext{##1}}}%
   \pagestyle{verticle}%
  }
  {\endantilandscape}

\makeatother

\newcommand\faketitle[1]{%
  \begingroup\trivlist
    \item\bfseries\Large #1
  \endtrivlist\endgroup}

\title{中文直排测试}
\date{}\author{Leo}

\begin{document}

\maketitle

鲁迅的一篇文章,只是例子而已。

\begin{CJKvert}

\faketitle{論“費厄潑賴”應該緩行\footnote{本篇最初發表于一九二六年一月十日《莽原》半月刊第一期。}}

\section{解題}


《語絲》五七期上語堂\footnote{林語堂(1895—1976):福建龍溪人,作家。早年留學美國、德國,曾任北京大學、北京女子師范大學教授,廈門大學文科主任,《語絲》撰稿人之一。當時与魯迅有交往,后因立場志趣日益歧异而斷交。三十年代,他在上海主編《論語》、《人間世》、《宇宙風》等雜志,以自由主義者的姿態,提倡“性靈”、“幽默”,為國民党反動統治粉飾太平。他在一九二五年十二月十四日《語絲》第五十七期發表《插論語絲的文体——穩健、罵人、及費厄潑賴》一文,其中說“‘費厄潑賴’精神在中國最不易得,我們也只好努力鼓勵,中國‘潑賴’的精神就很少,更談不到‘費厄’,惟有時所謂不肯‘下井投石’即帶有此義。罵人的人卻不可沒有這一樣條件,能駕人,也須能挨罵。且對于失敗者不應再施攻擊,因為我們所攻擊的在于思想非在人,以今日之段祺瑞、章士釗為例,我們便不應再攻擊其個人。”}先生曾經講起“費厄潑賴”(fair play)\footnote{“費厄潑賴”:英語 fair play 的音譯,原為体育比賽和其他競技所用的術語,意思是光明正大的比賽,不用不正當的手段。英國資產階級曾有人提倡將這种精神用于社會生活和党派斗爭中,認為這是每一個資產階級紳士應有的涵養和品德,并自稱英國是一個費厄潑賴的國度。但實際上,這不過是資產階級用以掩蓋自己的丑惡和麻痹人民群眾的一個漂亮口號。},以為此种精神在中國最不易得,我們只好努力鼓勵;又謂不“打落水狗”,即足以補充“費厄潑賴”的意義。我不懂英文,因此也不明這字的函義究竟怎樣,如果不“打落水狗”也即這种精神之一体,則我卻很想有所議論。但題目上不直書“打落水狗”者,乃為回避触目起見,即并不一定要在頭上強裝“義角”\footnote{“義角”:即假角。陳西瀅在《現代評論》第三卷五十三期(一九二五年十二月十二日)《閒話》中攻擊魯迅說:“花是人人愛好的,魔鬼是人人厭惡的。然而因為要取好于眾人,不惜在花瓣上加上顏色,在鬼頭上裝上義角,我們非但覺得無聊,還有些嫌它肉麻。”意思是說:魯迅的文章為讀者所歡迎,是因為魯迅為了討好讀者而假裝成一個戰斗者。}之意。總而言之,不過說是“落水狗”未始不可打,或者簡直應該打而已。

\section{論“落水狗”有三种,大都在可打之列}


今之論者,常將“打死老虎”与“打落水狗”相提并論,以為都近于卑怯\footnote{指吳稚暉、周作人、林語堂等人。吳稚暉在一九二五年十二月一日《京報副刊》發表的《官歟——共產党歟——吳稚暉歟》一文中說:現在批評章士釗,“似乎是打死老虎”。周作人在同月七日《語絲》五十六期的《失題》中則說:“打‘落水狗’(吾鄉方言,即‘打死老虎’之意)也是不大好的事。……一旦樹倒猢猻散,更從哪里去找這班散了的,況且在平地上追赶猢猻,也有點無聊卑劣。”林語堂在《插論語絲的文体——穩健、罵人、及費厄潑賴》一文中贊同周作人的意見,認為這正足以補充“‘費厄潑賴’的意義”。}。我以為“打死老虎”者,裝怯作勇,頗含滑稽,雖然不免有卑怯之嫌,卻怯得令人可愛。至于“打落水狗”,則并不如此簡單,當看狗之怎樣,以及如何落水而定。考落水原因,大概可有三种:(一)狗自己失足落水者,(二)別人打落者,(三)親自打落者。倘遇前二种,便即附和去打,自然過于無聊,或者竟近于卑怯;但若与狗奮戰,親手打其落水,則雖用竹竿又在水中從而痛打之,似乎也非已甚,不得与前二者同論。

听說剛勇的拳師,決不再打那已經倒地的敵手,這實足使我們奉為楷模。但我以為尚須附加一事,即敵手也須是剛勇的斗士,一敗之后,或自愧自悔而不再來,或尚須堂皇地來相報复,那當然都無不可。而于狗,卻不能引此為例,与對等的敵手齊觀,因為無論它怎樣狂嗥,其實并不解什么“道義”;況且狗是能浮水的,一定仍要爬到岸上,倘不注意,它先就聳身一搖,將水點洒得人們一身一臉,于是夾著尾巴逃走了。但后來性情還是如此。老實人將它的落水認作受洗,以為必已忏悔,不再出而咬人,實在是大錯而特錯的事。

總之,倘是咬人之狗,我覺得都在可打之列,無論它在岸上或在水中。

\section{論叭儿狗尤非打落水里,又從而打之不可}


叭儿狗一名哈吧狗,南方卻稱為西洋狗了,但是,听說倒是中國的特產,在万國賽狗會里常常得到金獎牌,《大不列顛百科全書》的狗照相上,就很有几匹是咱們中國的叭儿狗。這也是一种國光。但是,狗和貓不是仇敵么?它卻雖然是狗,又很像貓,折中,公允,調和,平正之狀可掬,悠悠然擺出別個無不偏激,惟獨自己得了“中庸之道”\footnote{“中庸之道”:儒家學說。《論語·雍也》:“中庸之為德也,其至矣乎!”宋代朱熹注:“中者,無過無不及之名;庸,平常也。……程子曰:‘不偏之謂中,不易之謂庸。中者,天下之正道,庸者,天下之定理。’”}似的臉來。因此也就為闊人,太監,太太,小姐們所鐘愛,种子綿綿不絕。它的事業,只是以伶俐的皮毛獲得貴人豢養,或者中外的娘儿們上街的時候,脖子上拴了細鏈于跟在腳后跟。

這些就應該先行打它落水,又從而打之;如果它自墜入水,其實也不妨又從而打之,但若是自己過于要好,自然不打亦可,然而也不必為之歎息。叭儿狗如可寬容,別的狗也大可不必打了,因為它們雖然非常勢利,但究竟還有些像狼,帶著野性,不至于如此騎牆。

以上是順便說及的話,似乎和本題沒有大關系。

\section{論不“打落水狗”是誤人子弟的}


總之,落水狗的是否該打,第一是在看它爬上岸了之后的態度。

狗性總不大會改變的,假使一万年之后,或者也許要和現在不同,但我現在要說的是現在。如果以為落水之后,十分可怜,則害人的動物,可怜者正多,便是霍亂病菌,雖然生殖得快,那性格卻何等地老實。然而醫生是決不肯放過它的。

現在的官僚和土紳士或洋紳士,只要不合自意的,便說是赤化,是共產;民國元年以前稍不同,先是說康党,后是說革党\footnote{康党:指曾經參加和贊成康有為等發動變法維新的人。革党,即革命党,指參加和贊成反清革命的人。},甚至于到官里去告密,一面固然在保全自己的尊榮,但也未始沒有那時所謂“以人血染紅頂子”\footnote{“以人血染紅頂子”:清朝官服用不同質料和顏色的帽頂子來區分官階的高低,最高的一品官是用紅寶石或紅珊瑚珠作帽頂子。清末的官僚和紳士常用告密和捕殺革命党人作為升官的手段,所以當時有“以人血染紅頂子”的說法。}之意。可是革命終于起來了,一群臭架子的紳士們,便立刻皇皇然若喪家之狗,將小辮子盤在頭頂上。革命党也一派新气,——紳士們先前所深惡痛絕的新气,“文明”得可以;說是“咸与維新”\footnote{“咸与維新”:語見《尚書·胤征》:“殲厥渠魁,脅從罔治,舊染污俗,咸与維新。”原意是對一切受惡習影響的人都給以棄舊從新的机會。這里指辛亥革命時革命派与反動勢力妥協,地主官僚等乘此投机的現象。}了,我們是不打落水狗的,听憑它們爬上來罷。于是它們爬上來了,伏到民國二年下半年,二次革命\footnote{二次革命:指一九一三年七月孫中山發動的討伐袁世凱的戰爭。与辛亥革命相對而言,故稱“二次革命”。在討袁軍發動之前和失敗之后,袁世凱曾指使他的走狗殺害了不少革命者。}的時候,就突出來幫著袁世凱咬死了許多革命人,中國又一天一天沉入黑暗里,一直到現在,遺老不必說,連遺少也還是那么多。這就因為先烈的好心,對于鬼蜮的慈悲,使它們繁殖起來,而此后的明白青年,為反抗黑暗計,也就要花費更多更多的气力和生命。

秋瑾\footnote{秋瑾(1879?—1907):字璇卿,號競雄,別號鑒湖女俠,浙江紹興人。一九〇四年留學日本,積极參加留日學生的革命活動,先后加入光复會、同盟會。一九〇六年春回國,一九〇七年在紹興主持大通師范學堂,組織光复軍,和徐錫麟准備在浙、皖兩省同時起義。徐錫麟起事失敗后,她于同年七月十三日被清政府逮捕,十五日凌晨被殺害于紹興軒亭口。
}女士,就是死于告密的,革命后暫時稱為“女俠”,現在是不大听見有人提起了。革命一起,她的故鄉就到了一個都督,——等于現在之所謂督軍,——也是她的同志:王金發\footnote{王金發(1882—1915):浙江嵊縣人,原是浙東洪門會党平陽党的首領,后加入光复會。辛亥革命后任紹興軍政分府都督,二次革命后于一九一五年七月被袁世凱的走狗浙江都督朱瑞殺害于杭州。}。他捉住了殺害她的謀主\footnote{謀主:据本文所述情節,是指當時紹興的大地主章介眉。他在作浙江巡撫增韞的幕僚時,极力慫恿掘毀西湖邊上的秋瑾墓。辛亥革命后因貪污納賄、平毀秋墓等罪被王金發逮捕,他用“捐獻”田產等手段獲釋。脫身后到北京任袁世凱總統府的秘書,一九一三年二次革命失敗后,他“捐獻”的田產即由袁世凱下令發還,不久他又參与朱瑞殺害王金發的謀划。按秋瑾案的告密者是紹興劣紳胡道南,他在一九〇八年被革命党人處死。},調集了告密的案卷,要為她報仇。然而終于將那謀主釋放了,据說是因為已經成了民國,大家不應該再修舊怨罷。但等到二次革命失敗后,王金發卻被袁世凱的走狗槍決了,与有力的是他所釋放的殺過秋瑾的謀主。

這人現在也已“壽終正寢”了,但在那里繼續跋扈出沒著的也還是這一流人,所以秋瑾的故鄉也還是那樣的故鄉,年复一年,絲毫沒有長進。從這一點看起來,生長在可為中國模范的名城\footnote{模范的名城:指無錫。陳西瀅在《現代評論》第二卷第三十七期(一九二五年八月二十二日)發表的《閒話》中說:“無錫是中國的模范縣”。}里的楊蔭榆\footnote{楊蔭榆(?—1938):江蘇無錫人,曾留學美國,一九二四年任北京女子師范大學校長。她依附北洋軍閥,壓迫學生,是當時推行帝國主義和封建主義的奴化教育的代表人物之一。}女士和陳西瀅先生,真是洪福齊天。

\section{論塌台人物不當与“落水狗”相提并論}


“犯而不校”\footnote{“犯而不校”:這是孔丘弟子曾參的話,見《論語·泰伯》。}是恕道,“以眼還眼以牙還牙”\footnote{“以眼還眼以牙還牙”:摩西的話,見《舊約·申命記》:“以眼還眼,以牙還牙,以手還手,以腳還腳。”}是直道。中國最多的卻是枉道:不打落水狗,反被狗咬了。但是,這其實是老實人自己討苦吃。

俗語說:“忠厚是無用的別名”,也許太刻薄一點罷,但仔細想來,卻也覺得并非唆人作惡之談,乃是歸納了許多苦楚的經歷之后的警句。譬如不打落水狗說,其成因大概有二:一是無力打;二是比例錯。前者且勿論;后者的大錯就又有二:一是誤將塌台人物和落水狗齊觀,二是不辨塌台人物又有好有坏,于是視同一律,結果反成為縱惡。即以現在而論,因為政局的不安定,真是此起彼伏如轉輪,坏人靠著冰山,恣行無忌,一旦失足,忽而乞怜,而曾經親見,或親受其噬嚙的老實人,乃忽以“落水狗”視之,不但不打,甚至于還有哀矜之意,自以為公理已伸,俠義這時正在我這里。殊不知它何嘗真是落水,巢窟是早已造好的了,食料是早經儲足的了,并且都在租界里。雖然有時似乎受傷,其實并不,至多不過是假裝跛腳,聊以引起人們的惻隱之心,可以從容避匿罷了。他日复來,仍舊先咬老實人開手,“投石下井”\footnote{“投石下井”:俗作“落井下石”,語出唐代韓愈的《柳子厚墓志銘》:“一旦臨小利害,僅如毛發,比反眼若不相識,落陷阱不一引手救,反擠之又下石焉者,皆是也。”林語堂在《插論語絲的文体——穩健、罵人、及費厄潑賴》一文中說:“不肯下井投石即帶有費厄潑賴之意”。},無所不為,尋起原因來,一部分就正因為老實人不“打落水狗”之故。所以,要是說得苛刻一點,也就是自家掘坑自家埋,怨天尤人,全是錯誤的。

\section{論現在還不能一味“費厄”}


仁人們或者要問:那么,我們竟不要“費厄潑賴”么?我可以立刻回答:當然是要的,然而尚早。這就是“請君入瓮”\footnote{“請君入瓮”:是唐朝酷吏周興的故事,見《資治通鑒》卷二〇四則天后天授二年:“或告文昌右丞周興与丘神崔通謀,太后命來俊臣鞫之,俊臣与興方推事對食,謂興曰:‘囚多不承,當為何法?’興曰:‘此甚易耳!取大瓮,以炭四周炙之,令囚入中,何事不承!’俊臣乃索大瓮,火圍如興法,因起謂興曰:‘有內狀推兄,請兄入此瓮!’興惶恐叩頭服罪。”}法。雖然仁人們未必肯用,但我還可以言之成理。土紳士或洋紳士們不是常常說,中國自有特別國情,外國的平等自由等等,不能适用么?我以為這“費厄潑賴”也是其一。否則,他對你不“費厄”,你卻對他去“費厄”,結果總是自己吃虧,不但要“費厄”而不可得,并且連要不“費厄”而亦不可得。所以要“費厄”,最好是首先看清對手,倘是些不配承受“費厄”的,大可以老實不客气;待到它也“費厄”了,然后再与它講“費厄”不遲。

這似乎很有主張二重道德之嫌,但是也出于不得已,因為倘不如此,中國將不能有較好的路。中國現在有許多二重道德,主与奴,男与女,都有不同的道德,還沒有划一。要是對“落水狗”和“落水人”獨獨一視同仁,實在未免太偏,太早,正如紳士們之所謂自由平等并非不好,在中國卻微嫌太早一樣。所以倘有人要普遍施行“費厄潑賴”精神,我以為至少須俟所謂“落水狗”者帶有人气之后。但現在自然也非絕不可行,就是,有如上文所說:要看清對手。而且還要有等差,即“費厄”必視對手之如何而施,無論其怎樣落水,為人也則幫之,為狗也則不管之,為坏狗也則打之。一言以蔽之:“党同伐异”\footnote{“党同伐异”:語見《后漢書·党錮傳序》。意思是糾合同伙,攻擊异己。陳西瀅曾在《現代評論》第三卷第五十三期(一九二五年十二月十二日)的《閒話》中用此語影射攻擊魯迅:“中國人是沒有是非的。……凡是同党,什么都是好的,凡是异党,什么都是坏的。”同時又標榜他們自己:“在‘党同伐异’的社會里,有人非但攻擊公認的仇敵,還要大膽的批評自己的朋友。”}而已矣。

滿心“婆理”\footnote{“婆理”:對“公理”而言,陳西瀅等人在女師大風潮中,竭力為楊蔭榆辯護,后又組織“教育界公理維持會”,反對女師大复校。這里所說的“紳士們”,即指他們。參看《華蓋集·“公理”的把戲》。}而滿口“公理”的紳士們的名言暫且置之不論不議之列,即使真心人所大叫的公理,在現今的中國,也還不能救助好人,甚至于反而保護坏人。因為當坏人得志,虐待好人的時候,即使有人大叫公理,他決不听從,叫喊僅止于叫喊,好人仍然受苦。然而偶有一時,好人或稍稍蹶起,則坏人本該落水了,可是,真心的公理論者又“勿報复”呀,“仁恕”呀,“勿以惡抗惡”呀……的大嚷起來。這一次卻發生實效,并非空嚷了:好人正以為然,而坏人于是得救。但他得救之后,無非以為占了便宜,何嘗改悔;并且因為是早已營就三窟,又善于鑽謀的,所以不多時,也就依然聲勢赫奕,作惡又如先前一樣。這時候,公理論者自然又要大叫,但這回他卻不听你了。

但是,“疾惡太嚴”,“操之過急”,漢的清流和明的東林\footnote{清流:指東漢末年的太學生郭泰、賈彪和大臣李膺、陳蕃等人。他們聯合起來批評朝政,暴露宦官集團的罪惡,于漢桓帝延熹九年(166)為宦官所誣陷,以結党為亂的罪名遭受捕殺,十余年間,先后四次被殺戮、充軍和禁錮的達七八百人,史稱“党錮之禍”。東林,指明末的東林党。主要人物有顧憲成、高攀龍等。他們聚集在無錫東林書院講學,議論時政,批評人物,對輿論影響很大。在朝的一部分比較正直的官吏,也和他們互通聲色,形成了一個以上層知識分子為主的政治集團。明天啟五年(1625)他們為宦官魏忠賢所屠殺,被害者數百人。},卻正以這一點傾敗,論者也常常這樣責備他們。殊不知那一面,何嘗不“疾善如仇”呢?人們卻不說一句話。假使此后光明和黑暗還不能作徹底的戰斗,老實人誤將縱惡當作寬容,一味姑息下去,則現在似的混沌狀態,是可以無窮無盡的。

\section[論“即以其人之道還治其人之身”]{論“即以其人之道還治其人之身”\footnote{“即以其人之道還治其人之身”:語見朱熹在《中庸》第十三章的注文。}}


中國人或信中醫或信西醫,現在較大的城市中往往并有兩种醫,使他們各得其所。我以為這确是极好的事。倘能推而廣之,怨聲一定還要少得多,或者天下竟可以臻于郅治。例如民國的通禮是鞠躬,但若有人以為不對的,就獨使他磕頭。民國的法律是沒有笞刑的,倘有人以為肉刑好,則這人犯罪時就特別打屁股。碗筷飯菜,是為今人而設的,有愿為燧人氏\footnote{燧人氏:我國傳說中最早鑽木取火的人,遠古的“三皇”之一。}以前之民者,就請他吃生肉;再造几千間茅屋,將在大宅子里仰慕堯舜的高士都拉出來,給住在那里面;反對物質文明的,自然更應該不使他銜冤坐汽車。這樣一辦,真所謂“求仁得仁又何怨”\footnote{“求仁得仁又何怨”:語見《論語·述而》。},我們的耳根也就可以清淨許多罷。

但可惜大家總不肯這樣辦,偏要以己律人,所以天下就多事。“費厄潑賴”尤其有流弊,甚至于可以變成弱點,反給惡勢力占便宜。例如劉百昭毆曳女師大學生\footnote{劉百昭:湖南武岡人,曾任北洋政府教育部專門教育司司長。一九二五年八月,章士釗解散女師大,另立女子大學,派劉百昭前往籌辦,劉于二十二日雇用流氓女丐毆打女師大學生,并將她們強拉出校。},《現代評論》上連屁也不放,一到女師大恢复,陳西瀅鼓動女大學生占据校舍時,卻道“要是她們不肯走便怎樣呢?你們總不好意思用強力把她們的東西搬走了罷?”\footnote{一九二五年十一月,女師大學生斗爭胜利,宣告复校,仍回原址上課。這時,陳西瀅在《現代評論》第三卷第五十四期(一九二五年十二月十九日)發表的《閒話》中,說了這里所引的話,鼓動女子大學學生占据校舍,破坏女師大复校。}毆而且拉,而且搬,是有劉百昭的先例的,何以這一回獨獨“不好意思”?這就因為給他嗅到了女師大這一面有些“費厄”气味之故。但這“費厄”卻又變成弱點,反而給人利用了來替章士釗的“遺澤”保鑣。

\section{結末}


或者要疑我上文所言,會激起新舊,或什么兩派之爭,使惡感更深,或相持更烈罷。但我敢斷言,反改革者對于改革者的毒害,向來就并未放松過,手段的厲害也已經無以复加了。只有改革者卻還在睡夢里,總是吃虧,因而中國也總是沒有改革,自此以后,是應該改換些態度和方法的。

一九二五年十二月二十九日。

\end{CJKvert}

回到正常横排文字。

\end{document}
