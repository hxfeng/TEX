\documentclass{article}
\usepackage{CJKutf8}
\newcommand{\chuhao}{\fontsize{42pt}{\baselineskip}\selectfont}
\newcommand{\xiaochuhao}{\fontsize{36pt}{\baselineskip}\selectfont}
\newcommand{\yihao}{\fontsize{28pt}{\baselineskip}\selectfont}
\newcommand{\erhao}{\fontsize{21pt}{\baselineskip}\selectfont}
\newcommand{\xiaoerhao}{\fontsize{18pt}{\baselineskip}\selectfont}
\newcommand{\sanhao}{\fontsize{15.75pt}{\baselineskip}\selectfont}
\newcommand{\sihao}{\fontsize{14pt}{\baselineskip}\selectfont}
\newcommand{\xiaosihao}{\fontsize{12pt}{\baselineskip}\selectfont}
\newcommand{\wuhao}{\fontsize{10.5pt}{\baselineskip}\selectfont}
\newcommand{\xiaowuhao}{\fontsize{9pt}{\baselineskip}\selectfont}
\newcommand{\liuhao}{\fontsize{7.875pt}{\baselineskip}\selectfont}
\newcommand{\qihao}{\fontsize{5.25pt}{\baselineskip}\selectfont}
\begin{document}
\begin{CJK}{UTF8}{gbsn}
这是一个CJK例子,使用了UTF-8编码和gbsn字体。使用pdflatex编译即可


在印刷出版上,中文字号制与点数制的对照关系如下:\\

 1770年法国人狄道(F.A.Didot)制定点数制,规定1法寸为72点,即:1点=0.3759毫米。

 狄道点数制在法国、德国、奥地利、比利时、丹麦、匈牙利等国比较流行。

 1886年全美活字铸造协会以派卡(pica)为基准制定派卡点数制,规定1pica=12point(点),即:

 \fbox{1点=0.013837英寸=0.35146毫米}\\

 20世纪初派卡点数制传入我国,并得到逐步推广。在实用中对常用点数以号数命名而产生了号数制,

 二者换算如下(以pt代表“点”):\\

 \begin{center}

 \begin{tabular}{r@{\ =\ }l}

 初号& 42pt\\

 小初号& 36pt\\

 一号& 28pt\\

 二号& 21pt\\

 小二号& 18pt\\

 三号& 15.75pt\\

 四号& 14pt\\

 小四号& 12pt\\

 五号& 10.5pt\\

 小五号& 9pt\\

 六号 & 7.875pt\\

 七号 & 5.25pt

 \end{tabular}

 \end{center}


具体示例如下:
\\
\\
\chuhao{初号小苹果}\\
\xiaochuhao{小初号小苹果}\\
\yihao{一号小苹果}\\
\erhao{二号小苹果}\\
\xiaoerhao{小二号小苹果}\\
\sanhao{三号小苹果}\\
\sihao{四号小苹果}\\
\xiaosihao{小四号小苹果}\\
\wuhao{五号小苹果}\\
\xiaowuhao{小五号小苹果}\\
\liuhao{六号小苹果}\\
\qihao{七号小苹果}\\

\end{CJK}
\end{document}
