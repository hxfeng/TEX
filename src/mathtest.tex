\documentclass[paper=a4]{scrartcl}
\usepackage[T1]{fontenc}
\usepackage[utf8]{inputenc}
\usepackage{booktabs}
\usepackage{amsmath}
\title{Calcul théorique de la propulsion éléctromagnetique de l'aimant}
\author{Lawrence}
\begin{document}
\maketitle
\section*{Constantes ``modifiables''}
\begin{table}[h]
\begin{tabular}{lc}
\toprule
Rayon des spires & \(D\)\\
Nombre de spire & \(N\)\\
Tension du générateur & \(\mu\)\\
Capacité du condensateur & \(C\)\\
Arètes du tore & \(L\)\\
Masse du tore & \(m\)\\
\bottomrule
\end{tabular}
\end{table}
\section*{Variables}
\begin{table}[h]
\begin{tabular}{lc}
\toprule
Vitesse de tore & \(v(t)\)\\
Tension du condensateur & \(u(t)\)\\
Intensité & \(i(t)\)\\
Champ magnétique & \(\vec{B}\)\\
Potentiel vecteur & \(\vec{A}\)\\
Densité de courant dans le tore & \(\vec{j}\)\\
\bottomrule
\end{tabular}
\end{table}
\section*{Equations}
\begin{equation}
\boxed{
m \frac{dv}{dt} = -mg + \iiint_{
\textrm{tore-en-mouvement}} (\vec{j} \times \vec{B}) \cdot \vec{e_z
}d\tau
}
\label{m}
\end{equation}
\begin{equation}
i = -C \frac{du}{dt}
\label{i}
\end{equation}
\begin{equation}
u = Ri + \iint_{N-spires} \vec{B} \cdot d\vec{S}
\label{u}
\end{equation}
\begin{equation}
\vec{j} = \gamma \left(
\vec{v} \times \vec{B} - \frac{\partial\vec{A}}{\partial t}
\right)
\label{j}
\end{equation}
\begin{equation}
\vec{A} = \frac{\mu_0}{4\pi} \iiint_{tore-en-mouvement} \frac{\vec{j}}{PM}
\cdot d\tau + \frac{\mu_0 Ni}{4\pi} \int\frac{d\vec{r}}{PM}
\label{A}
\end{equation}
\begin{equation}
\vec{B} = rot\vec{A}
\label{B}
\end{equation}
\end{document}
