\documentclass{article}

\usepackage[T1]{fontenc}

\usepackage{fix-cm}
\usepackage{CJKutf8}
\newcommand{\chuhao}{\fontsize{42pt}{\baselineskip}\selectfont}
\newcommand{\xiaochuhao}{\fontsize{36pt}{\baselineskip}\selectfont}
\newcommand{\yihao}{\fontsize{28pt}{\baselineskip}\selectfont}
\newcommand{\erhao}{\fontsize{21pt}{\baselineskip}\selectfont}
\newcommand{\xiaoerhao}{\fontsize{18pt}{\baselineskip}\selectfont}
\newcommand{\sanhao}{\fontsize{15.75pt}{\baselineskip}\selectfont}
\newcommand{\sihao}{\fontsize{14pt}{\baselineskip}\selectfont}
\newcommand{\xiaosihao}{\fontsize{12pt}{\baselineskip}\selectfont}
\newcommand{\wuhao}{\fontsize{10.5pt}{\baselineskip}\selectfont}
\newcommand{\xiaowuhao}{\fontsize{9pt}{\baselineskip}\selectfont}
\newcommand{\liuhao}{\fontsize{7.875pt}{\baselineskip}\selectfont}
\newcommand{\qihao}{\fontsize{5.25pt}{\baselineskip}\selectfont}
\begin{document}
\begin{CJK}{UTF8}{gbsn}
这是一个CJK例子,使用了UTF-8编码和gbsn字体。使用pdflatex编译即可


在印刷出版上,中文字号制与点数制的对照关系如下:\\

 1770年法国人狄道(F.A.Didot)制定点数制,规定1法寸为72点,即:1点=0.3759毫米。

 狄道点数制在法国、德国、奥地利、比利时、丹麦、匈牙利等国比较流行。

 1886年全美活字铸造协会以派卡(pica)为基准制定派卡点数制,规定1pica=12point(点),即:

 \fbox{1点=0.013837英寸=0.35146毫米}\\

 20世纪初派卡点数制传入我国,并得到逐步推广。在实用中对常用点数以号数命名而产生了号数制,

 二者换算如下(以pt代表“点”):\\

 \begin{center}

 \begin{tabular}{r@{\ =\ }l}

 初号& 42pt\\

 小初号& 36pt\\

 一号& 28pt\\

 二号& 21pt\\

 小二号& 18pt\\

 三号& 15.75pt\\

 四号& 14pt\\

 小四号& 12pt\\

 五号& 10.5pt\\

 小五号& 9pt\\

 六号 & 7.875pt\\

 七号 & 5.25pt

 \end{tabular}

 \end{center}


具体示例如下:
\\
\\
\chuhao{初号小苹果}\\
\xiaochuhao{小初号小苹果}\\
\yihao{一号小苹果}\\
\erhao{二号小苹果}\\
\xiaoerhao{小二号小苹果}\\
\sanhao{三号小苹果}\\
\sihao{四号小苹果}\\
\xiaosihao{小四号小苹果}\\
\wuhao{五号小苹果}\\
\xiaowuhao{小五号小苹果}\\
\liuhao{六号小苹果}\\
\qihao{七号小苹果}\\
\begin{figure}
\begin{center}
% GNUPLOT: LaTeX picture
\setlength{\unitlength}{0.240900pt}
\ifx\plotpoint\undefined\newsavebox{\plotpoint}\fi
\sbox{\plotpoint}{\rule[-0.200pt]{0.400pt}{0.400pt}}%
\begin{picture}(1500,1200)(0,0)
\sbox{\plotpoint}{\rule[-0.200pt]{0.400pt}{0.400pt}}%
\put(231.0,131.0){\rule[-0.200pt]{4.818pt}{0.400pt}}
\put(211,131){\makebox(0,0)[r]{$0$}}
\put(1430.0,131.0){\rule[-0.200pt]{4.818pt}{0.400pt}}
\put(231.0,320.0){\rule[-0.200pt]{4.818pt}{0.400pt}}
\put(211,320){\makebox(0,0)[r]{$0.2$}}
\put(1430.0,320.0){\rule[-0.200pt]{4.818pt}{0.400pt}}
\put(231.0,509.0){\rule[-0.200pt]{4.818pt}{0.400pt}}
\put(211,509){\makebox(0,0)[r]{$0.4$}}
\put(1430.0,509.0){\rule[-0.200pt]{4.818pt}{0.400pt}}
\put(231.0,699.0){\rule[-0.200pt]{4.818pt}{0.400pt}}
\put(211,699){\makebox(0,0)[r]{$0.6$}}
\put(1430.0,699.0){\rule[-0.200pt]{4.818pt}{0.400pt}}
\put(231.0,888.0){\rule[-0.200pt]{4.818pt}{0.400pt}}
\put(211,888){\makebox(0,0)[r]{$0.8$}}
\put(1430.0,888.0){\rule[-0.200pt]{4.818pt}{0.400pt}}
\put(231.0,1077.0){\rule[-0.200pt]{4.818pt}{0.400pt}}
\put(1430.0,1077.0){\rule[-0.200pt]{4.818pt}{0.400pt}}
\put(231.0,131.0){\rule[-0.200pt]{0.400pt}{4.818pt}}
\put(231,90){\makebox(0,0){$0$}}
\put(231.0,1057.0){\rule[-0.200pt]{0.400pt}{4.818pt}}
\put(425.0,131.0){\rule[-0.200pt]{0.400pt}{4.818pt}}
\put(425,90){\makebox(0,0){$1$}}
\put(425.0,1057.0){\rule[-0.200pt]{0.400pt}{4.818pt}}
\put(619.0,131.0){\rule[-0.200pt]{0.400pt}{4.818pt}}
\put(619,90){\makebox(0,0){$2$}}
\put(619.0,1057.0){\rule[-0.200pt]{0.400pt}{4.818pt}}
\put(813.0,131.0){\rule[-0.200pt]{0.400pt}{4.818pt}}
\put(813,90){\makebox(0,0){$3$}}
\put(813.0,1057.0){\rule[-0.200pt]{0.400pt}{4.818pt}}
\put(1007.0,131.0){\rule[-0.200pt]{0.400pt}{4.818pt}}
\put(1007,90){\makebox(0,0){$4$}}
\put(1007.0,1057.0){\rule[-0.200pt]{0.400pt}{4.818pt}}
\put(1202.0,131.0){\rule[-0.200pt]{0.400pt}{4.818pt}}
\put(1202,90){\makebox(0,0){$5$}}
\put(1202.0,1057.0){\rule[-0.200pt]{0.400pt}{4.818pt}}
\put(1396.0,131.0){\rule[-0.200pt]{0.400pt}{4.818pt}}
\put(1396,90){\makebox(0,0){$6$}}
\put(1396.0,1057.0){\rule[-0.200pt]{0.400pt}{4.818pt}}
\put(231.0,131.0){\rule[-0.200pt]{0.400pt}{227.891pt}}
\put(231.0,131.0){\rule[-0.200pt]{293.657pt}{0.400pt}}
\put(1450.0,131.0){\rule[-0.200pt]{0.400pt}{227.891pt}}
\put(231.0,1077.0){\rule[-0.200pt]{293.657pt}{0.400pt}}
\put(70,604){\makebox(0,0){This is the $y$ axis}}
\put(840,29){\makebox(0,0){This is the $x$ axis}}
\put(231,131){\usebox{\plotpoint}}
\multiput(231.58,131.00)(0.492,2.564){21}{\rule{0.119pt}{2.100pt}}
\multiput(230.17,131.00)(12.000,55.641){2}{\rule{0.400pt}{1.050pt}}
\multiput(243.58,191.00)(0.493,2.360){23}{\rule{0.119pt}{1.946pt}}
\multiput(242.17,191.00)(13.000,55.961){2}{\rule{0.400pt}{0.973pt}}
\multiput(256.58,251.00)(0.492,2.521){21}{\rule{0.119pt}{2.067pt}}
\multiput(255.17,251.00)(12.000,54.711){2}{\rule{0.400pt}{1.033pt}}
\multiput(268.58,310.00)(0.492,2.478){21}{\rule{0.119pt}{2.033pt}}
\multiput(267.17,310.00)(12.000,53.780){2}{\rule{0.400pt}{1.017pt}}
\multiput(280.58,368.00)(0.493,2.281){23}{\rule{0.119pt}{1.885pt}}
\multiput(279.17,368.00)(13.000,54.088){2}{\rule{0.400pt}{0.942pt}}
\multiput(293.58,426.00)(0.492,2.392){21}{\rule{0.119pt}{1.967pt}}
\multiput(292.17,426.00)(12.000,51.918){2}{\rule{0.400pt}{0.983pt}}
\multiput(305.58,482.00)(0.492,2.349){21}{\rule{0.119pt}{1.933pt}}
\multiput(304.17,482.00)(12.000,50.987){2}{\rule{0.400pt}{0.967pt}}
\multiput(317.58,537.00)(0.493,2.122){23}{\rule{0.119pt}{1.762pt}}
\multiput(316.17,537.00)(13.000,50.344){2}{\rule{0.400pt}{0.881pt}}
\multiput(330.58,591.00)(0.492,2.176){21}{\rule{0.119pt}{1.800pt}}
\multiput(329.17,591.00)(12.000,47.264){2}{\rule{0.400pt}{0.900pt}}
\multiput(342.58,642.00)(0.492,2.133){21}{\rule{0.119pt}{1.767pt}}
\multiput(341.17,642.00)(12.000,46.333){2}{\rule{0.400pt}{0.883pt}}
\multiput(354.58,692.00)(0.492,2.004){21}{\rule{0.119pt}{1.667pt}}
\multiput(353.17,692.00)(12.000,43.541){2}{\rule{0.400pt}{0.833pt}}
\multiput(366.58,739.00)(0.493,1.765){23}{\rule{0.119pt}{1.485pt}}
\multiput(365.17,739.00)(13.000,41.919){2}{\rule{0.400pt}{0.742pt}}
\multiput(379.58,784.00)(0.492,1.789){21}{\rule{0.119pt}{1.500pt}}
\multiput(378.17,784.00)(12.000,38.887){2}{\rule{0.400pt}{0.750pt}}
\multiput(391.58,826.00)(0.492,1.659){21}{\rule{0.119pt}{1.400pt}}
\multiput(390.17,826.00)(12.000,36.094){2}{\rule{0.400pt}{0.700pt}}
\multiput(403.58,865.00)(0.493,1.408){23}{\rule{0.119pt}{1.208pt}}
\multiput(402.17,865.00)(13.000,33.493){2}{\rule{0.400pt}{0.604pt}}
\multiput(416.58,901.00)(0.492,1.444){21}{\rule{0.119pt}{1.233pt}}
\multiput(415.17,901.00)(12.000,31.440){2}{\rule{0.400pt}{0.617pt}}
\multiput(428.58,935.00)(0.492,1.272){21}{\rule{0.119pt}{1.100pt}}
\multiput(427.17,935.00)(12.000,27.717){2}{\rule{0.400pt}{0.550pt}}
\multiput(440.58,965.00)(0.493,1.012){23}{\rule{0.119pt}{0.900pt}}
\multiput(439.17,965.00)(13.000,24.132){2}{\rule{0.400pt}{0.450pt}}
\multiput(453.58,991.00)(0.492,0.970){21}{\rule{0.119pt}{0.867pt}}
\multiput(452.17,991.00)(12.000,21.201){2}{\rule{0.400pt}{0.433pt}}
\multiput(465.58,1014.00)(0.492,0.841){21}{\rule{0.119pt}{0.767pt}}
\multiput(464.17,1014.00)(12.000,18.409){2}{\rule{0.400pt}{0.383pt}}
\multiput(477.58,1034.00)(0.493,0.616){23}{\rule{0.119pt}{0.592pt}}
\multiput(476.17,1034.00)(13.000,14.771){2}{\rule{0.400pt}{0.296pt}}
\multiput(490.58,1050.00)(0.492,0.539){21}{\rule{0.119pt}{0.533pt}}
\multiput(489.17,1050.00)(12.000,11.893){2}{\rule{0.400pt}{0.267pt}}
\multiput(502.00,1063.59)(0.758,0.488){13}{\rule{0.700pt}{0.117pt}}
\multiput(502.00,1062.17)(10.547,8.000){2}{\rule{0.350pt}{0.400pt}}
\multiput(514.00,1071.59)(1.378,0.477){7}{\rule{1.140pt}{0.115pt}}
\multiput(514.00,1070.17)(10.634,5.000){2}{\rule{0.570pt}{0.400pt}}
\put(527,1075.67){\rule{2.891pt}{0.400pt}}
\multiput(527.00,1075.17)(6.000,1.000){2}{\rule{1.445pt}{0.400pt}}
\multiput(539.00,1075.95)(2.472,-0.447){3}{\rule{1.700pt}{0.108pt}}
\multiput(539.00,1076.17)(8.472,-3.000){2}{\rule{0.850pt}{0.400pt}}
\multiput(551.00,1072.93)(0.874,-0.485){11}{\rule{0.786pt}{0.117pt}}
\multiput(551.00,1073.17)(10.369,-7.000){2}{\rule{0.393pt}{0.400pt}}
\multiput(563.00,1065.92)(0.652,-0.491){17}{\rule{0.620pt}{0.118pt}}
\multiput(563.00,1066.17)(11.713,-10.000){2}{\rule{0.310pt}{0.400pt}}
\multiput(576.58,1054.65)(0.492,-0.582){21}{\rule{0.119pt}{0.567pt}}
\multiput(575.17,1055.82)(12.000,-12.824){2}{\rule{0.400pt}{0.283pt}}
\multiput(588.58,1040.09)(0.492,-0.755){21}{\rule{0.119pt}{0.700pt}}
\multiput(587.17,1041.55)(12.000,-16.547){2}{\rule{0.400pt}{0.350pt}}
\multiput(600.58,1021.90)(0.493,-0.814){23}{\rule{0.119pt}{0.746pt}}
\multiput(599.17,1023.45)(13.000,-19.451){2}{\rule{0.400pt}{0.373pt}}
\multiput(613.58,1000.13)(0.492,-1.056){21}{\rule{0.119pt}{0.933pt}}
\multiput(612.17,1002.06)(12.000,-23.063){2}{\rule{0.400pt}{0.467pt}}
\multiput(625.58,974.71)(0.492,-1.186){21}{\rule{0.119pt}{1.033pt}}
\multiput(624.17,976.86)(12.000,-25.855){2}{\rule{0.400pt}{0.517pt}}
\multiput(637.58,946.50)(0.493,-1.250){23}{\rule{0.119pt}{1.085pt}}
\multiput(636.17,948.75)(13.000,-29.749){2}{\rule{0.400pt}{0.542pt}}
\multiput(650.58,913.74)(0.492,-1.487){21}{\rule{0.119pt}{1.267pt}}
\multiput(649.17,916.37)(12.000,-32.371){2}{\rule{0.400pt}{0.633pt}}
\multiput(662.58,878.47)(0.492,-1.573){21}{\rule{0.119pt}{1.333pt}}
\multiput(661.17,881.23)(12.000,-34.233){2}{\rule{0.400pt}{0.667pt}}
\multiput(674.58,841.35)(0.493,-1.607){23}{\rule{0.119pt}{1.362pt}}
\multiput(673.17,844.17)(13.000,-38.174){2}{\rule{0.400pt}{0.681pt}}
\multiput(687.58,799.64)(0.492,-1.832){21}{\rule{0.119pt}{1.533pt}}
\multiput(686.17,802.82)(12.000,-39.817){2}{\rule{0.400pt}{0.767pt}}
\multiput(699.58,756.22)(0.492,-1.961){21}{\rule{0.119pt}{1.633pt}}
\multiput(698.17,759.61)(12.000,-42.610){2}{\rule{0.400pt}{0.817pt}}
\multiput(711.58,710.33)(0.493,-1.924){23}{\rule{0.119pt}{1.608pt}}
\multiput(710.17,713.66)(13.000,-45.663){2}{\rule{0.400pt}{0.804pt}}
\multiput(724.58,660.67)(0.492,-2.133){21}{\rule{0.119pt}{1.767pt}}
\multiput(723.17,664.33)(12.000,-46.333){2}{\rule{0.400pt}{0.883pt}}
\multiput(736.58,610.39)(0.492,-2.219){21}{\rule{0.119pt}{1.833pt}}
\multiput(735.17,614.19)(12.000,-48.195){2}{\rule{0.400pt}{0.917pt}}
\multiput(748.58,557.97)(0.492,-2.349){21}{\rule{0.119pt}{1.933pt}}
\multiput(747.17,561.99)(12.000,-50.987){2}{\rule{0.400pt}{0.967pt}}
\multiput(760.58,503.56)(0.493,-2.162){23}{\rule{0.119pt}{1.792pt}}
\multiput(759.17,507.28)(13.000,-51.280){2}{\rule{0.400pt}{0.896pt}}
\multiput(773.58,447.70)(0.492,-2.435){21}{\rule{0.119pt}{2.000pt}}
\multiput(772.17,451.85)(12.000,-52.849){2}{\rule{0.400pt}{1.000pt}}
\multiput(785.58,390.56)(0.492,-2.478){21}{\rule{0.119pt}{2.033pt}}
\multiput(784.17,394.78)(12.000,-53.780){2}{\rule{0.400pt}{1.017pt}}
\multiput(797.58,333.05)(0.493,-2.320){23}{\rule{0.119pt}{1.915pt}}
\multiput(796.17,337.02)(13.000,-55.025){2}{\rule{0.400pt}{0.958pt}}
\multiput(810.58,273.28)(0.492,-2.564){21}{\rule{0.119pt}{2.100pt}}
\multiput(809.17,277.64)(12.000,-55.641){2}{\rule{0.400pt}{1.050pt}}
\multiput(822.58,213.42)(0.492,-2.521){21}{\rule{0.119pt}{2.067pt}}
\multiput(821.17,217.71)(12.000,-54.711){2}{\rule{0.400pt}{1.033pt}}
\multiput(834.59,154.99)(0.485,-2.399){11}{\rule{0.117pt}{1.929pt}}
\multiput(833.17,159.00)(7.000,-27.997){2}{\rule{0.400pt}{0.964pt}}
\put(231.0,131.0){\rule[-0.200pt]{0.400pt}{227.891pt}}
\put(231.0,131.0){\rule[-0.200pt]{293.657pt}{0.400pt}}
\put(1450.0,131.0){\rule[-0.200pt]{0.400pt}{227.891pt}}
\put(231.0,1077.0){\rule[-0.200pt]{293.657pt}{0.400pt}}
\end{picture}

\end{center}
\end{figure}
\end{CJK}
\end{document}
