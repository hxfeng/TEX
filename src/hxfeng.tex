%% start of file `template-zh.tex'.
%% Copyright 2006-2013 Xavier Danaux (xdanaux@gmail.com).
%
% This work may be distributed and/or modified under the
% conditions of the LaTeX Project Public License version 1.3c,
% available at http://www.latex-project.org/lppl/.


\documentclass[11pt,a4paper,sans]{moderncv}   % possible options include font size ('10pt', '11pt' and '12pt'), paper size ('a4paper', 'letterpaper', 'a5paper', 'legalpaper', 'executivepaper' and 'landscape') and font family ('sans' and 'roman')
\usepackage[slantfont,boldfont]{xeCJK} %使用xeCJK宏包
\setCJKmainfont{WenQuanYi Micro Hei Mono}
\setCJKmainfont[BoldFont=Adobe Heiti Std,ItalicFont=Adobe Kaiti Std]{Adobe Song Std} %设置正文为宋体,粗体使用黑体,斜体使用楷体
\setCJKmonofont{Adobe Song Std} %设置等距字体
\setCJKsansfont[BoldFont=Adobe Heiti Std]{Adobe Kaiti Std} %设置无衬线字体
%%-----------------------设置英文字体-----------------------------------------------------------------------------------------
\setmainfont[Mapping=tex-text]{TeX Gyre Pagella Math} %英文衬线字体
\setsansfont[Mapping=tex-text]{Trebuchet MS}     %英文无衬线字体
\setmonofont[Mapping=tex-text]{Courier}      %英文等宽字体
%%%%----------重定义中文字体%%%%---------------
\setCJKfamilyfont{song}{Adobe Song Std}
\setCJKfamilyfont{hei}{Adobe Heiti Std}
\setCJKfamilyfont{kai}{Adobe Kaiti Std}
\setCJKfamilyfont{fs}{Adobe Fangsong Std}
%%-------------------------------------
\newcommand{\song}{\CJKfamily{song}}
\newcommand{\hei}{\CJKfamily{hei}}
\newcommand{\kai}{\CJKfamily{kai}}
\newcommand{\fs}{\CJKfamily{fs}}
\punctstyle{kaiming} %开明式标点格式
\linespread{1.5}   % 1.5倍行距
% moderncv 主题
\moderncvstyle{oldstyle}                        % 选项参数是 ‘casual’, ‘classic’, ‘oldstyle’ 和 ’banking’
\moderncvcolor{blue}                          % 选项参数是 ‘blue’ (默认)、‘orange’、‘green’、‘red’、‘purple’ 和 ‘grey’
\nopagenumbers{}                             % 消除注释以取消自动页码生成功能

% 字符编码
%\usepackage[utf8]{inputenc}                   % 替换你正在使用的编码
\usepackage{xeCJK}

% 调整页面出血
\usepackage[scale=0.75]{geometry}
%\setlength{\hintscolumnwidth}{3cm}           % 如果你希望改变日期栏的宽度

% 个人信息
\name{zhang}{san}
%\title{个人简历}                     % 可选项、如不需要可删除本行
\address{福州市仓山区金山大道}{福建省福州市}            % 可选项、如不需要可删除本行
\phone[mobile]{1234567890}              % 可选项、如不需要可删除本行
%\phone[fixed]{+2~(345)~678~90}               % 可选项、如不需要可删除本行
%\phone[fax]{+3~(456)~789~012}                 % 可选项、如不需要可删除本行
\email{hxfeng987@163.com}                    % 可选项、如不需要可删除本行
%\homepage{www.xialongli.com}                  % 可选项、如不需要可删除本行
%\extrainfo{附加信息 (可选项)}                 % 可选项、如不需要可删除本行
%\photo[64pt][0.4pt]{picture}                  % ‘64pt’是图片必须压缩至的高度、‘0.4pt‘是图片边框的宽度 (如不需要可调节至0pt)、’picture‘ 是图片文件的名字;可选项、如不需要可删除本行
%\quote{引言(可选项)}                          % 可选项、如不需要可删除本行

% 显示索引号;仅用于在简历中使用了引言
\makeatletter
\renewcommand*{\bibliographyitemlabel}{\@biblabel{\arabic{enumiv}}}
\makeatother

% 分类索引
\usepackage{multibib}
\newcites{book,misc}{{Books},{Others}}
%----------------------------------------------------------------------------------
%            内容
%----------------------------------------------------------------------------------
\begin{document}
%\begin{CJK}{UTF8}{gbsn}                       % 详情参阅CJK文件包
\maketitle
\section{基本信息}
\cvdoubleitem{姓名}{zhang san }{性别}{男}
\cvdoubleitem{电话}{1234567890}{邮箱}{hxfeng987@163.com}
\cvdoubleitem{民族}{汉族}{政治面貌}{共产党员}

\section{教育背景}
\cventry{2009年 到 2013年}{学士学位}{西北农林科技大学}{中国杨凌}{\textit{优秀}}{}  % 第3到第6编码可留白
%\cventry{年 -- 年}{学位}{院校}{城市}{\textit{成绩}}{说明}
\cvitem{在校专业}{信息与计算科学(计算数学)}
%\section{毕业论文}
%\cvitem{题目}{\emph{题目}}
%\cvitem{导师}{导师}
%\cvitem{说明}{\small 论文简介}
\section{求职岗位}
数据分析师
\section{工作背景}
\subsection{软件开发工程师}
\cventry{2013年 至今 }{c++软件工程师}{星网锐捷}{福州市}{}{主要负责数据中心的应用程序开发\newline{}%
工作内容:%
\begin{itemize}%
\item 数据中心服务器端应用程序开发
\item 大数据分析处理 
 % \begin{itemize}%
 % \item 独立安装配置CDH系列软件,开发与数据中心相关的数据挖掘应用程序,使用的工具软件有hive、impala、hbase等
 % \item 使用splunk软件,使用splunk软件进行日志查询分析以及相关的计算最后通过报表展示数据分析结果。
%\item 曾经使用cacti进行系统监控。
% (b)、含三级内容 (不建议使用);
 %   \begin{itemize}
  %  \item 三级内容 i;
   % \item 三级内容 ii;
    %\item 三级内容 iii;
   % \end{itemize}
 % \item 维护公司linux服务器正常运行,涉及到的服务器系统种类有centos ,redhat 等
 % \end{itemize}
%\item 数据中心RabbitMQ的基本维护和相关的开发
\end{itemize}}
%\cventry{年 -- 年}{职位}{公司}{城市}{}{说明行1\newline{}说明行2}
%\subsection{其他}
%\cventry{年 -- 年}{职位}{公司}{城市}{}{说明}

\section{语言技能}
\cvitemwithcomment{英语}{四级}{具有良好的听说读写能力,经常翻阅英文资料,可以直接阅读英文资料应用实际工作当中。}
\cvitemwithcomment{日语}{初级}{可以使用简单的日常用语进行日常交流}
%\cvitemwithcomment{语言 3}{水平}{评价}

\section{计算机技能}
%\cvdoubleitem
\cvitem{数据库}{熟悉sql语句熟练使用mysql}\cvitem{c/c++}{擅长linux环境下的c/c++开发}

%\cvdoubleitem
\cvitem{python}{熟悉python应用开发(非web方向)}\cvitem{latex}{熟练掌握latex的排版}
%\cvdoubleitem
\cvitem{web开发}{了解web开发的基本内容}\cvitem{办公软件}{熟练使用办公软件}
%\cvdoubleitem
\cvitem{操作系统}{熟悉windows及linux主要发行版本如centos ubuntu等}\cvitem{QT和MFC}{熟悉QT和MFC开发基本框架}
%\cvdoubleitem
\cvitem{linux脚本}{熟悉使用shell脚本及linux下其他相关工具如sed awk perl 等}%\cvitem{UML建模}{掌握系统的的UML建模知识}
%\cvdoubleitem
\cvitem{数学软件}{熟悉matlab,mathematics,maple等数学软件}\cvitem{R语言}{了解R语言的基本知识,可以进行数据处理和分析}
\section{项目经验}
\cvitem{仓库管理系统}
 {使用MFC进行的开发,通过jdbc驱动链接数据,项目主要是通过MFC开发出的界面管理数据库中的仓库数据。}
\cvitem{大数据数据收集系统}
{通过收集系统产生的日志,上传到分布式文件系统,将分布式系统中的数据转换成hive的数据元,通过impala进行实时查询计算,最后将有必要存储的结果存入mysql数据交给前端用php展示数据。}
\section{个人兴趣}
%\cvitem{看    书}{\small 书是让人不堕落的最好的伙伴,读书让我更充实快乐。}
\cvitem{互联网学习}{\small 现在的互联网非常的发达,在互联网上获得的知识远远大于在学校学到的知识,当今社会互联网才是最好的大学。}
%\cvitem{钻研技术}{\small 我喜欢研究新的东西,酷爱钻研技术,尤其喜欢如数学和计算机相关的东西。}

%\section{其他}
%\cvlistitem{一件事物的品质在于其内在本质,与别人的贬低和赞美没有丝毫关系。}
%\cvlistitem{学习机器语言让与机器交流,学习人类的语言与人类交流}
%\cvlistitem{要获得从未有过的成功,我就要付出从未有过的努力,所以我一之在坚持努力}

\renewcommand{\listitemsymbol}{-}             % 改变列表符号

%\section{其他 2}
%\cvlistdoubleitem{项目 1}{项目 4}
%\cvlistdoubleitem{项目 2}{项目 5\cite{book1}}
%\cvlistdoubleitem{项目 3}{}

% 来自BibTeX文件但不使用multibib包的出版物
%\renewcommand*{\bibliographyitemlabel}{\@biblabel{\arabic{enumiv}}}% BibTeX的数字标签
\nocite{*}
\bibliographystyle{plain}
\bibliography{publications}                    % 'publications' 是BibTeX文件的文件名

% 来自BibTeX文件并使用multibib包的出版物
%\section{出版物}
%\nocitebook{book1,book2}
%\bibliographystylebook{plain}
%\bibliographybook{publications}               % 'publications' 是BibTeX文件的文件名
%\nocitemisc{misc1,misc2,misc3}
%\bibliographystylemisc{plain}
%\bibliographymisc{publications}               % 'publications' 是BibTeX文件的文件名

\clearpage
%\end{CJK}
\end{document}


%% 文件结尾 `template-zh.tex'.
