\documentclass{article}
\usepackage{indentfirst}%关于缩进
\usepackage[colorlinks,bookmarksnumbered,bookmarksopen,linkcolor=red]{hyperref}
\usepackage{geometry}%关于页面设置

\geometry{left=2.5cm,right=2.5cm,top=2.5cm,bottom=2.5cm}
%\usepackage{hyperref}%关于交叉引用
%\usepackage{ref}
\usepackage[BoldFont,SlantFont,CJKchecksingle]{xeCJK}
\setCJKmainfont[BoldFont=SimHei]{SimSun}
\setCJKmonofont{SimSun}% 设置缺省中文字体

\begin{document}
\section{this is fouth section}
用罗马字符显示当前的section次序\Roman{section}
%\arabic{section}
\renewcommand\thesection{\Roman{section}}改变计数器显示的样式为罗马字符
\setcounter{section}{4}当前改变计数器为4
\setcounter{page}{3}改变页码为3
\section{this is fouth section is it? no this is five section}
i want use a footnot at here
%\footnote{hell}
\footnotetext[1]{first foot note}添加了一个注脚
\footnotemark
hahah one more again
\footnotetext[2]{second foot note}添加了第二个注脚
\footnotemark
this is footnote itself\footnote{helloworld}添加了第三个注脚
我们看看TeX怎么在文中加脚注的. 一般格式是:
\label{hello}\  footnote[num]\{内容\}
\ref{hello}
其中[num]是可选项, 如果省略, 则自动给该页脚注排序, 从1开
始标号. footnote只能用于普通正文模式中, 比如盒子或数学环
境中都无法使用, 因此系统提供了另外两条命令:
\ footnotemark[num] 用来在文中插入脚注符号, 但不产生脚注内
容; 而 \  footnotetext[num]\{内容\} 则用来真正产生脚注内容. 注
意在以后要讲的minipage中, 系统可能会将脚注放错位置. 
\section{greetings}
\label{sec:greetings}交叉引用首先设置标签
 然后这里可以写其他内容。。。。。。。。。。。。。。。。。。。。。。。。。。。。。。。。。。。。。。。。。。。。。。。。。。。。。。。。。。。。。。。。。。。。。。。。。。。。。。。。。。。。。。。。。。。。。。。。。。。

\section{新的section}
 
I greeted in section \ref{sec:greetings}这里是引用greetings的那个section可以自动调转
\begin{equation} \label{eq:solve}
x^2 - 5 x + 6 = 0
\end{equation}
 
\begin{equation}
x_1 = \frac{5 + \sqrt{25 - 4 \times 6}}{2} = 3
\end{equation}
 
\begin{equation}
x_2 = \frac{5 - \sqrt{25 - 4 \times 6}}{2} = 2
\end{equation}
 
and so we have solved equation \ref{eq:solve}


 \begin{equation}
 \lim_{x \to 0}\frac{\sin x}{x}=1
 \label{eq:myequation}
 \end{equation}
(\ref{eq:myequation})式是一个很重要的极限.



\begin{center}
居中文字 \

居中文字

\end{center}
 

\noindent

\begin{minipage}{\linewidth}

\centering
居中文字 \

居中文字

\end{minipage}
 

\begin{flushleft}
左对齐文字 \

左对齐文字

\end{flushleft}
 

\noindent 取消缩进

\begin{minipage}{\linewidth}
最小页开始
\raggedright
左对齐文字 \

左对齐文字

\end{minipage}
 

\begin{flushright}
右对齐文字 \

右对齐文字

\end{flushright}
 

\noindent

\begin{minipage}{\linewidth}

\raggedleft
右对齐文字 \

右对齐文字

\end{minipage}

行间距

%\linespread{因子}
或者
%\renewcommand{\baselinestretch}{因子}

比如

\linespread{1.5}
或者
\renewcommand{\baselinestretch}{1.5}
设置行间距为1.5倍\\
是不是呢\\
段间距

%设置\parskip的值,比如

\setlength{\parskip}{0.5\baselineskip}

首行缩进

如果默认首行不缩进,则使用indentfirst宏包

%\usepackage{indentfirst}

指定某段首行缩进,在段首加

\indent

指定某段首行不缩进,在段首加

\noindent

设置缩进量

\setlength\parindent{2em}

悬挂缩进

在要悬挂缩进的段落前加上

\noindent
\hangafter=1
\setlength{\hangindent}{2em}

分别是: 1.取消首行缩进;2. 设置从第1行之后开始悬挂缩进;3. 设置悬挂缩进量

\end{document}
